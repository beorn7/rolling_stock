\documentclass[10pt,final]{report}
\usepackage[utf8]{inputenc}
\usepackage[letterpaper,
            top=1.8cm, bottom=1.8cm, left=1.6cm, right=1.6cm]{geometry}
\usepackage{multicol,graphicx}
\usepackage{sfmath,mathabx,wasysym}
\usepackage[table]{xcolor}
\usepackage[compact]{titlesec}

%\renewcommand{\familydefault}{\sfdefault}

%\setlength{\parindent}{0pt}
%\setlength{\parskip}{0.3ex plus 0.5ex minus 0.2ex}
\setlength{\columnseprule}{0pt}
\setlength{\columnsep}{1.8cm}



%\titleformat{\chapter}[display]
%{\normalfont\bfseries}
%{\LARGE Chapter \thechapter}
%{1ex}
%{\LARGE}
%[{}]

\definecolor{red}{rgb}{1.0,0.3,0.3}
\definecolor{orange}{rgb}{1.0,0.7,0.2}
\definecolor{yellow}{rgb}{1.0,1.0,0.0}
\definecolor{green}{rgb}{0.5,1.0,0.5}
\definecolor{blue}{rgb}{0.7,0.7,1.0}
\definecolor{purple}{rgb}{0.8,0.5,1.0}

\begin{document}

%\title{This page to be replaced by custom title page,
%indicating that chapter 1 of this booklet is the first
%thing a new player should read.}
%\author{Björn Rabenstein}
%\date{v1.2sp (\today)}
%\maketitle

\textbf{TODO: Decide if this \LaTeX source should be refined into the
  final version, or everything should just be imported into AAG's
  usual DTP program.}

\textbf{TODO: Add title page.}

\textbf{TODO: Replace \$ by the final money symbol.}

\textbf{TODO: Adjust to final game design (corporation names and
  symbols (or call them logo now?), description of components, does
  play money exist?, circles vs. diamonds for synergies \dots).}

\textbf{TODO: Perhaps move certain things into ``boxes'': Receivership
  could be explained in a box rather than its own section. Whenever a
  special ability is explained, it could be put into a box with the
  corporations logo in the background.}

\setcounter{tocdepth}{1}
\setcounter{page}{3}
\tableofcontents

\chapter{Learning the game}

\begin{multicols}{2}

  This chapter is an introduction to the game \emph{Rolling Stock
    Stars}. It is written in a casual and easy-to-understand way, with
  a lot of examples, figures, and helpful explanations. Read it first
  to learn the game.

  Later, when you need to refresh your knowledge of the rules or you
  want a precise answer to rules questions, refer to the canonical
  rules (the smaller booklet delivered with the game). Those rules
  are supposed to be the single and complete ``source of
  truth''. Thus, all the examples and explanations in this
  introduction are meant as a help to understand the rules but not to
  add or change any rules.

\section{Overview}

\emph{Rolling Stock Stars} is a card game for three to six
players. The players take the role of investors. They buy private
companies, which they may later turn into corporations or sell to
already existing corporations. In addition, they can trade shares of
those corporations. The player with the most shares in a corporation
becomes its president and controls its actions. Corporations may own
any number of those formerly private companies (that were sold to them
by players or were used as the seed to found a new
corporation). Companies owned by corporations are called
\emph{subsidiary companies}. Corporations can even buy subsidiary
companies from each other.

Note the nomenclature here: Corporations could also be called ``public
companies''. However, to avoid confusion, we want to keep the meaning
of the word ``company'' narrow and will not use it for corporations
(but for the subsidiary companies they own).

Subsidiary companies owned by the same corporation may create
synergies with each other, increasing the income of the
corporation. These synergies can be seen as a quite abstract
representation of transportation networks. (\emph{Rolling Stock Stars}
has no board and therefore lacks a more concrete representation of
transportation networks, as you might know it from similarly themed
games.)

As more and more newer companies are brought into the game,
older companies become less profitable and have to be written off
eventually. Corporations have to struggle to stand the test of time.

In the end, the richest player wins the game, measured by the added
values of privately owned companies, shares of corporations, and cash.

Thematically, the game starts in the 1830s in Prussia. The private
companies initially available are early Prussian railroad
companies. Throughout the game, the scope widens. First to Germany,
then to Europe. Even some non-railroad companies entere the game; the
last tier of companies represents container ports and airports.

\section{Components}

The game contains two booklets: One is titled \emph{Player's Guide}
(you are reading it right now), the other is the actual rules (written
in a very concise and formal way -- comes in handy if you are looking
for precise answers to your most delicate rules questions, but
definitely not suitable for learning the rules).

There are six \emph{turn summaries}, describing the nine phases of
a game turn. Hand out one to each player. 

\textbf{TODO: Add figure with ``synergy markers''.}

Furthermore, you will find 109 round double-sided \emph{synergy
  markers} (see figure above). There is something missing in the box,
however: \emph{play money}.  Please use play money from another game
or --~much preferred~-- poker chips. \textbf{TODO: If play money is
  included, change this sentence.} The synergy markers and the money
are meant to be unlimited. In the unlikely case that you run out of
these components, find other means of tracking money and synergies.
(Even a small set of poker chips should easily be enough for the game,
most likely you will not need more than 100 \$1 chips, 100 \$5 chips,
and 50 \$25 chips. As a last resort, you might even use real coins as
play money, 1 ``real'' cent translates into \$1 in the game.)

\emph{Rolling Stock Stars} is a card game, so there are obviously cards, 131
of them. Let's look at them in detail.

\subsection{Player order cards}

\textbf{TODO: Add figure ``player order cards''.}

There are 6 \emph{player order cards}. They are used to randomly
determine the initial player order and later track the player order
throughout the game.

\subsection{Company cards}

\textbf{TODO: Add figure ``company cards''.}

\textbf{TODO: Adjust to actual layout of company cards.}

The 36 \emph{company cards} are the core of the game. They come in five
colors. To assist color-blind players, each color has a geometric
shape assigned to it: Red ($\bullet$), orange ($\blacktriangleup$),
yellow ($\sqbullet$), green ($\pentagon$), blue ($\hexagon$).

Each company has a unique face value, printed in the upper left
corner. The face value is used when calculating the wealth
of a player at the end of the game (see section \ref{win}). It is also
the mimimum bid in auctions (see section \ref{t1p1}), and since it is
unique, it can be used to identify the company (like a serial number).

The numbers printed in parentheses to the right of the face value
define the price span the company can be sold for to
corporations. Just beneath the face value, there are a number of stars
(one to five), which are used to determine share price changes of a
corporation (section \ref{t2p6}).

In the upper right corner, you'll find a circle with a
``+\$'' amount. This is the income of of the company. In the middle
between the face value and the income, you can see the name of the
company and its abbreviation. Name, abbreviation, and face value are
each unique. So you can refer to the \emph{Société nationale des
  chemins de fer français} as ``the Société nationale des chemins de
fer français'' or ``the SNCF'' or ``the 24''. Whatever you like
most. The colorful boxes on the company card tell you something about
the possible synergies with other companies. They will be explained
later (section \ref{t2p5}).

The back of each company card shows a \emph{cost of ownership}
(starting from no cost up to \$4). The cost of ownership
printed on the back of a company card has nothing to do with the cost
of ownership of the company described on the face of the same
card. When you set up the game, you will build a deck of face-down
company cards. The back of the top-most card in that deck determines
the current cost of ownership applying to all companies whose color
matches one of the colors in the central rectangle on the back of the
card. A more detailed explanation of the cost of ownership will follow
later (section \ref{t3p5}).

\subsection{Game end card}

\textbf{TODO: Add figure ``game end card''.}

The colorful card that have a cost of ownership on both sides is the
\emph{game end card} (\$7 and \$10). The game end card is used as the
bottom-most card of the company card deck.

\subsection{Symbol cards}

\textbf{TODO: Add figure ``symbol cards''.}

There are 8 \emph{symbol cards}. Each features the symbol, name, and
color of a corporation in a central box, together with the text ``All
X Shares Issued'' (where X is a number between 4 and 7). A
symbol card is the central component of the corresponding
corporation. Its shares (see below) are put into the central box. The
money it owns is placed right of the symbol card, and its share price
card (see below) to the left. In a horizontal row below the symbol
card, you line up all the companies the corporation owns as
\emph{subsidiaries} (which might be as few as one).

\subsection{Share cards}

\textbf{TODO: Add figure ``share cards''.}

There are 44 share cards, between 4 and 7 for each corporation. The
shares are represented as smaller cards, featuring the symbol and
color of the respective corporation. The shares are numbered, with the
1\textsuperscript{st} share marked as the \emph{president's} share.

\subsection{Share price cards}

\textbf{TODO: Add figure ``share price cards''.}

There are 27 white share price cards. These cards are used to mark the
current price of each share of a corporation. They show the share
price in the center. Left of the share price, some share price cards
feature an IPO box. Right of the share price, you'll find another box
showing you the maximum payout per share. The upper left and upper
right corner show the two next lower or next higher share prices,
respectively. The list cross-referencing numbers of shares with
numbers of stars is used for share price adjustments, as explained
later.

\subsection{Receivership cards}

\textbf{TODO: Add figure ``receivership card''.}

There are 8 cards titled \emph{corporation in receivership}. They are
used to mark corporations that are in receivership and contain a rule
summary how those corporations act in the various phases of a turn.

\subsection{Foreign investor card}

\textbf{TODO: Add figure ``foreign investor card''.}

Finally there is the \emph{foreign investor}. The card contains quite
a lot of text, explaining the actions of the foreign investor, a kind
of dummy player. The card is also used to arrange the assets of the
foreign investor, similar to the symbol card of corporations. Its
money goes to the right of the card, its companies in a horizontal row
below it.

The rules will often prompt you to ``turn a card vertically'' or to
``turn a card back horizontally''. The standard (and ``default'')
orientation of a card is called ``horizontal'' (i.\,e. if the rules
don't state anything else, a card is oriented horizontally). If you
turn the card by 90 degrees, its orientation is called
``vertical''. Vertical orientation marks a special state of a card and
is mentioned explicitly in the rules wherever it applies. (Note that
the player order cards are printed in ``portrait'' orientation rather
than the usual ``landscape'' orientation. Still their normal
orientation is referred to as ``horizontal'' and their ``special''
orientation as ``vertical''.)

\section{Setting up the game}

Set up the game following these steps:

\begin{enumerate}
\item Each player should have a turn summary handy throughout the
  game.
\item Place the money in a central position on the table, easily
  reachable for everybody. This central area is the
  \emph{bank}. Initially, it contains all the money in the game, but
  later, it will also contain shares.
\item Give each player \$30 from the bank. \emph{Exception:} In a
  six-player game, each player receives only \$25.
\item Pick the player order cards corresponding to the number of
  players. Return the remaining cards to the box. Shuffle the player
  order cards and deal one random card to each player. The players
  reveal their cards, which define the initial player order. You don't
  need to change seating order as the player order will change often
  throughout the game.
\item Set the eight symbol cards aside, separately. On top of each
  symbol card, into the central box, place the shares of the
  corresponding corporation. Sort the shares, with the
  1\textsuperscript{st} share on top.
\item In another area of the table, lay out the 27 share price cards in a
  long, sorted row, starting with \$0 and ending with \$75. Most
  tables will not be long enough for this row. Feel free to break the
  row, e.\,g. into three rows of nine cards each. But keep in mind
  that it is effectively still one long row.
\item Pick one player who will be in charge of executing the actions
  of the foreign investor. (There are no decisions involved. That
  player only has to make sure that those actions are executed
  according to the rules and not forgotten.) Place the foreign
  investor card in reach of that player. Place \$4 (from the bank)
  into the treasury of the foreign investor (i.\,e. to the right of
  the foreign investor card).
\item In the following steps, you'll build the company deck. Start
  with the game end card. Place it on the table where it is easily
  visible for all players (somewhere next to the bank). Turn the face
  with the lower cost of ownership (\$7) up.
\item Sort the company cards by color and set aside the company with
  the highest face value of each color. That is the CDG (60) for blue,
  the E (43) for green, the DR (29) for yellow, the PR (19) for
  orange, and the MHE (8) for red.
\item Shuffle each remaining pile of companies of the same color.
\item Without looking at them, draw a number of companies from each
  pile that is equal to the number of players. \emph{Exceptions: With
    4 players, draw 5 orange companies. With 5 players, draw 7 orange
    companies. With 6 players, simply use all companies of all
    colors.} Keep the drawn companies separated by color.
\item Return the remaining companies to the box, again without looking
  at them.
\item Shuffle each of the set-aside companies into the pile of drawn
  companies of the same color.
\item Now build the deck by placing the blue companies on top of the
  game end card, then the green companies on top of the blue
  companies, then the yellow on top of green, then the orange on top
  of yellow, and finally the red on top of orange.
\item From the deck, draw and reveal a number of company cards equal
  to the number of players. Place them next to the deck. These
  companies are now in the \emph{offering}. They are all available for
  auctions in the first turn of the game.
\end{enumerate}

You are all set to start the first turn of the game.

\textbf{TODO: Add figure ``set up''.}

\section{The first turn}

Each turn runs through nine phases (although in some turns, nothing
might happen in particular phases). Refer to the turn summary to get
an overview. The right-most column of the turn summary indicates who
makes decisions in a particular phase: \emph{PRIV} means that the
players act as private investors. \emph{CORP} means that the players
act as presidents of the corporations. (The president of a corporation
is the player that currently holds the president's share of that
corporation.) \emph{AUTO} means that no decisions are required. The
game ``plays itself'' in those phases.

\subsection{Phase 1 -- Investment}
\label{t1p1}

In current player order, starting with position 1, each player performs
exactly one action. However, the player order is cyclic, so after the
player last in player order has taken their action, loop back to
player 1, who will now take exactly one action again. Proceed with
player 2, and so on. Repeat this cycle until you meet the end
condition described below.

In turn 1, the only actions available are \emph{pass} and \emph{start
  an auction}. (The share trading part of this phase is still missing
in turn 1.)

\emph{Pass} is a very simple action: If you take that action, you
(basically) do nothing. As a reminder that your last action was
\emph{pass}, you turn your player order card vertically. Passing does
not prevent you from taking a different action next time you are up.
If you take an action different from \emph{pass}, but you have passed
before, turn your player order card back horizontally.

If at any time during this phase, all player order cards are turned
vertically, the phase immediately ends. In other words, the phase
ends once all players have passed consecutively. Even if passing
itself does not prevent you from taking another action next time you
are up (see above), the phase might end before you have the
opportunity to do so (which happens if everybody else passes, too).

\emph{Start an auction} is the other possible action. If you take that
action, you pick one of the companies available for auction. You place
a bid at least as high as the face value of that company. If you don't
have enough money to do so, or if there is no company available for
auction, you cannot take this action. Once you have placed your bid,
the next player in player order either raises the bid by at least \$1
or leaves the auction. Then the next player does the same, and so
on. Remember that the player order is cyclic, so after the last player
in player order has raised the bid or left the auction, the player at
position 1 is up to either raise the bid or leave the auction. This
cycle continues until all players but one have left the auction.  The
remaining player pays their bid to the bank and places the company
card in front of them. A player's bid must not exceed the money the
player owns (but it may exceed the price range printed on the company
card as that range is only relevant for acquisitions in phase
3). Players that have left the auction are skipped for the remainder
of that auction. They are not allowed to re-enter that same auction.

Once the auction is over, a new company card is drawn and placed
face-up into the offering of companies. However, the newly drawn card
is not available for auctions during the same phase it is drawn. Turn
the company card vertically to mark it as unavailable. In the first
few turns of the game, it is very common that at some point during
this phase all the companies in the offering are not available for
auctions so that players cannot start more auctions.

The player that takes the next action after the auction is the one
next in player order after the player who \emph{started the auction}
(\emph{not} after the player who won the auction).

\emph{Example of a complete auction: Alice, Bob, and Chris play a
  three-player game. They have already reached turn 2. (The auctions
  in turn 1 are less interesting, so this example is taken from turn
  2. The rules are exactly the same.) The current player order is
  Alice: 1, Bob: 2, Chris: 3. Alice has \$20, Bob \$12, Chris \$9. The
  offering contains the MHE, the WT, and the MS. The WT has been
  drawn this turn, so it is oriented vertically. Bob is up to pick an
  action. He wants to start an auction. However, the WT is not
  available for auctions, and the MS is too expensive (minimum bid is
  \$17 but Bob has only \$12). The only company Bob could pick is the
  MHE. He does so and decides to place an initial bid of \$9. \$8
  would have been a legal bid, too, but Bob wanted to kick Chris out
  of the auction from the start. Chris is next in player order but has
  only \$9 so he cannot raise the bid and automatically leaves the
  auction. Next is Alice. She has enough money to raise the bid. She
  decides to raise the bid to \$11. Now it's back to Bob. He would
  still be able to raise the bid to \$12 but he thinks that \$12 for
  the MHE is a bit too much. Furthermore, if he leaves now, Alice has
  won the auction and has to pay the \$11 she has bid. After that, she
  won't have enough money to buy the MS, which she would have gotten
  for face value otherwise because no other player would have had
  enough money to overbid her. So Bob leaves the auction, Alice pays
  \$11 and gets the MHE. A new company is drawn, the BD, which is
  placed into the offering, but turned vertically. The auction is over
  now, and the next player to take an action is Chris (because he is
  next in player order after Bob, who started the auction). Chris
  doesn't have enough money to start an auction. So he has to pass and
  turns his player order card vertically. In fact, the only company
  available for auctions is the MS, and none of the players have enough
  money left to bid for it, so all players have to pass, and the phase
  ends.}

\subsection{Phase 2 -- Wrap-up}

Redistribute player order cards according to remaining cash on
hand. The player with the most cash left gets position 1, and so
on. Break ties using the old player order. (In practice, you should
first check if there are any ties, break them according to the current
distribution of player order cards, and only then start to
redistribute the player order cards.)

\emph{Example (continuing the example above): After
  the end of phase 3, both Alice and Chris have \$9 left. Bob has \$12
  left. So Bob will be on position 1 in the new player order. Alice
  and Chris tie for position 2. Since Alice was before Chris in the
  old player order (1 vs. 3), she gets position 2, and Chris keeps
  position 3.}

If there are any companies left that are available for auction, the
foreign investor tries to buy them for face value directly (no auction
triggered), starting with the company with the lowest face value and
then continuing in ascending face value order. If the foreign investor
has enough money for the company with the lowest face value, it pays
it to the bank and adds the company to its assets. (Place the company
below the foreign investor card. If the foreign investor already owns
companies, line them up in a horizontal row.) Then repeat with the
company with the next lowest face value, and so on, until the foreign
investor doesn't have enough money left.

Whenever the foreign investor buys a company, draw a new one from the
deck as if that company had been purchased in an auction.

Usually, it takes a few turns before the foreign investor manages to
buy a company. In practice, it is very rare that it buys more than one
company in one turn.

After the foreign investor is done, turn all companies in the offering
horizontally, so that they are available for auctions in phase 1 of the
next turn.

\subsection{Phase 3 -- Acquisition}

As there are no corporations in the game yet, nothing happens in this
phase.

\subsection{Phase 4 -- Closing}

In principle, you could close your freshly bought companies already,
but it really wouldn't make any sense. So ignore this phase for now.

\subsection{Phase 5 -- Income}

The bank pays income to all players and to the foreign investor. Each
player adds the income of all their companies and collects the result
from the bank. (The income of each company is printed in the circle in
the upper right corner of the company card.)

The foreign investor does the same, but always earns an additional
+\$5 bonus, regardless of owning any companies (see the circle in the
upper right corner of the foreign investor card). Add the foreign
investor's income to its treasury to the right of the foreign investor
card.

\subsection{Phase 6 -- Dividends}

Only corporations pay dividends. As you probably have guessed by now,
there are no corporations yet, so nothing happens.

\subsection{Phase 7 -- End card}

Only once we approach the end of the game, something will happen in
this phase. Ignore it for now.

\subsection{Phase 8 -- Issue shares}

We are nearly there, but at the moment, we still have no corporations
in the game. So once again, nothing to see here.

\subsection{Phase 9 -- IPO}

Players that own private companies can decide to go public with one or
more of them, i.\,e. convert them into \emph{corporations}. Only
companies owned by players can be converted into corporations (but not
companies owned by the foreign investor or by already existing
corporations). In descending face value order, the owners of the
eligible companies decide if they want to go public or not. If they go
public, the whole procedure is completed for that company before the
owner of the next company decides.

\emph{Example: Alice owns the MS (face value \$17) and the KME
  (\$5). Bob owns the WT (\$11) and the BPM (\$7). Chris owns the BSE
  (\$2). The first company that may go public is the MS. Alice has to
  decide first, and cannot revise her decision later during the same
  phase. Once she has decided if the MS goes public (and if so, has
  performed the required procedure), Bob decides for the WT and then
  for the BPM. After that, Alice decides for the KME, and finally
  Chris decides for the BSE.}

The conversion procedure is the following:
\begin{enumerate}
\item Pick one of the symbol cards that is currently not in
  use. Together with the pile of shares on top, place it a bit away
  from your personal assets (it's going to be a \emph{public} company
  after all) but still in your reach. (In the unlikely case that there
  is no unused symbol card available, your company cannot be
  converted. Sorry.) Note that each corporation has a different
  special ability, and a varying amount of shares to issue (from four
  to seven). We'll cover the special abilities later, when they become
  relevant.
\item Place the company that is being converted below the symbol card.
\item From the share price cards that are currently not in use, choose
  an eligible starting price for your corporation. The eligible share
  price cards are those that feature the color of the company being
  converted in their IPO box. Place the chosen share price card left
  of the symbol card. (That will leave an empty spot in the row of
  share price cards. Leave it alone, don't move the other share price
  cards to close the gap.) The share price card determines the current
  value of each share of the corporation.  In turn 2, all privately
  owned companies are red, so the allowed starting share prices are
  \$10, \$11, \$12, \$13, and \$14. However, in later turns, players
  will own differently colored companies, too, which changes the range
  of eligible starting share prices for those companies. (It is even
  possible that all eligible share price cards are in use by other
  corporations. In that case, your company cannot be
  converted. Sorry.)
\item Now take the first share from the stack of shares (which is the
  golden president's share) and place it in front of you. That's now
  your share, which you got in exchange for the company you went
  public with.
\item As the share you have received has a higher share price than the
  face value of the company you went public with, you have to pay the
  difference from your private money into the treasury of the
  corporation. Place the money to the right of the share price
  card. (If you don't have enough money to pay the difference, the
  whole procedure is void and has to be undone. If possible, you can
  choose a lower share price so that you have enough money to pay the
  difference between share price and face value of your company. But
  if not, you cannot convert your company. Sorry once more.)
\item The whole point of going public is to get public investors. So
  next, you place the second share from the stack into the bank. In
  return, the bank pays the share price into the treasury of the
  corporation. (This step is mandatory. You cannot opt to not give a
  share to the bank.)
\end{enumerate}

\textbf{TODO: Add figure illustrating the example below.}

\emph{Example: You go public with the MHE (\$8 face value). You pick
  the ``Jupiter'' corporation (which has a special ability that is
  especially helpful for the early red companies, we'll get to that
  later).  As starting share price, you choose \$11. You have to pay
  \$3 into the treasury of your newly formed corporation. You receive
  the president's share of the Jupiter in return. The bank receives the
  second Jupiter share and pays \$11 into the treasury of the
  corporation. Arrange all involved components as shown in the figure
  above. The Jupiter corporation consists of the \$11 share price card,
  the Jupiter symbol card with the two remaining shares in the central
  box, the company card of the MHE, and \$14 cash in its treasury.}

\section{The second turn}

Congratulations. You have finished your first turn. The second turn is
going to become a bit more interesting. Some of the phases won't be
ignored any longer, and others get more complex.

\subsection{Phase 1 -- Investment}

This phase works the same as in turn 1, but now we will add share
trading to the mix. There are two more possible actions to chose from:
\emph{buy one share} and \emph{sell one share}.

Only shares owned by the bank can be bought. (The pile of shares on
top of the symbol cards cannot be touched in this phase.) If you choose
the \emph{buy one share} action, perform the following steps:
\begin{enumerate}
\item Take the desired share from the bank and place it in front of you.
\item Return the share price card of the corresponding corporation to
  its place in the row of share price cards and replace it by the next
  higher available share price card. (Usually, that is the next higher
  card as marked by the single arrow in the upper right corner of the
  share price card. \textbf{TODO: Adjust to actual card design.}
  However, if that share price card is in use, you will skip it. It is
  even possible that many consecutive share price cards are used and
  the share price of the share you are currently buying jumps up a
  lot.)
\item Now pay the \emph{new} share price to the bank. (If you don't
  have enough money to do so, the whole action is void. Undo
  everything, and try something else.)
\end{enumerate}

Now re-read the last item in that list and think about the
consequences. Like in real share trading, the displayed share price of
a corporation is not the price you have to pay if you want to buy one
of the shares. It is more like the ``last known share price''. By
buying a share, you are already modifying the system, and you have to
pay more than that ``last known share price''.

After you have bought a share of a corporation you are not the
president of, check if you now own more shares of that corporation
than the current president. If that is the case, you have managed
something like a hostile takeover. \emph{You} are now the president of
that corporation. Exchange the golden president's share with any one
of your shares of the same corporation. (Shares are basically
all the same. The numbering only matters as long as they are
still on the stack on top of the symbol card. The golden president's
share, however, is used as a marker for the current
president. In all other regards, it's a perfectly normal share.)

If you choose the \emph{sell one share} action, perform the following
steps:
\begin{enumerate}
\item Choose one of your shares to sell and place it into the
  bank. (Only choose the golden president's share if it is the last
  share you own of that corporation.)
\item Return the share price card of the corresponding corporation to
  its place in the row of share price cards and replace it by the next
  lower available share price card. (Usually, that is the next lower
  card as marked by the single arrow in the upper left corner of the
  share price card. \textbf{TODO: Adjust to actual card design.}
  However, as before, you will skip missing share price cards.)
\item Now the bank pays you the \emph{new} share price.
\end{enumerate}

Similar to the situation when buying a share, the displayed share
price is not the price you get paid. Like buying, selling a share
modifies the system, so you get less money for a share than its ``last
known value''.

Another similarity is that a change of presidency might occur after
the transaction: If you are the current president, and you have sold a
share, you have to check if now another player (\emph{not} the bank)
owns more shares of that corporation than you. That player exchanges
one of their shares of that corporation with the golden president's
share (which might still be owned by you, or it is owned by the bank
if you have just sold it as your last share of that corporation). In
the case where more than one player is tied for the most shares, the
player that is following you closer in player order becomes the
president.

\emph{Example: The player order is Chris: 1, Alice: 2, Bob: 3. The
  ``Eagle'' corporation has currently three shares issued.  (That
  cannot be the case in turn 2, only in later turns, you'll see.)
  Alice is the president and owns the golden president's share. Chris
  and Bob own each one of the other two shares. It's Alice's turn to
  take one action. She decides to sell one ``Eagle'' share, following
  the procedure above. After that, both, Chris and Bob own more shares
  each than Alice. As they are tied, the player order decides who
  becomes the new president. Alice is followed by Bob, and Bob is
  followed by Chris (remember, the player order is cyclic, the last
  player is followed by the first). So Bob is following Alice closer
  than Chris and becomes the new president. He takes the president's
  share from the bank and places his own share into the bank in
  exchange. Now Bob owns the president's share, and Chris and the bank
  own one share each. Alice no longer owns an ``Eagle'' share.}

But what happens if you, as the president, sell your last share (the
golden president's share), and \emph{no other player} owns any shares
of the corporation? Or in other words: What happens if the bank owns
\emph{all} issued shares of a corporation? In that special case, the
corporation goes \emph{into receivership}, which we'll cover in its
own section~\ref{rec} below.

\subsection{Phase 2 -- Wrap-up}

This phase works exactly the same as in turn 1.

\subsection{Phase 3 -- Acquisition}

In this phase, corporations may buy companies. Only corporations may
buy, but they may buy from anybody: players, the foreign investor, and
even other corporations (but not from the offering of companies
available for auctions -- those are indeed only available for players
in phase 1, and for the foreign investor in phase 2). In phase 3,
players and the foreign investor only \emph{sell} companies, never
buy.

This is the first time where the presidency of a corporation becomes
relevant. The president of a corporation decides on behalf of the
corporation. It might easily happen that both sides of a deal are
actually controlled by the same player. If you (as a player) own a
company and you are at the same time the president of a corporation,
there is nothing wrong if you (as a player) agree with yourself (as
the president of the corporation) that the corporation will buy your
company for a price you agree on with yourself.

In every single transaction, exactly one company is bought. Buyer and
seller have to agree on a price within the price span printed on the
company card. (This price span is inclusive, e.\,g. the allowed prices
for the KME (\$5 face value) are \$3, \$4, \$5, \$6, and \$7.) The
buying corporation must be able to pay the price from its
treasury. (You cannot pay with other means, or buy ``bundles''. It's
always one company for an allowed amount of cash.)

Any number of transactions might happen during the phase, in any
desired order, even concurrently. Think of a marketplace. Buyers and
sellers find each other at will, by announcing their offers to
whomever they want, negotiating in all directions. And once a buyer
and a seller agree on a deal, they make it happen. The phase goes on
until no transactions are happening any longer. There are some
restrictions, though:
\begin{itemize}
\item Every \$ and every company may only be part of one single
  transaction in the whole phase. The money that has been paid is
  turned vertically, as well as the company that has been handed over,
  to mark those components as ``in flight''. They cannot be part of
  another transaction in the same phase. Once the phase is over, you
  can turn them all horizontally again. They have ``arrived'' by then
  and can be used normally. (If you use poker chips, you'll find it
  hard to recognize chips that have been ``turned
  vertically''. Instead, place the chips used in a transaction on top
  of the stack of unissued shares.)
\item At any time, each corporation must own at least one subsidiary
  company. You can never completely ``empty'' a corporation. This one
  company might very well be a company turned vertically, i.\,e. a
  corporation that owns only one subsidiary company in the beginning
  of the phase could first buy another company (which is thereby turned
  vertically) and then sell the company it originally owned. There is
  no hierarchy of subsidiary companies within a corporations. The
  company a corporation owned when it was formed has no special
  status within the corporation.
\end{itemize}

A special case is the foreign investor, as nobody controls it and its
decisions. The foreign investor will happily sell any company it owns,
but only for the maximum allowed price. The money it receives if it
sells a company goes into its treasury and is available to be used in
phase 2 of later turns. There might still be an ambiguous situation if
more than one corporation want to buy the same company from the
foreign investor. In that case, the corporation with the higher share
price card has priority. In practice, whenever a corporation wants to
buy a company from the foreign investor, its president has to announce
the intention. At that time, pause the game for a short while and ask
each president of a corporation with a higher share price if they want
to intervene and buy that company immediately themself. If more than
one corporation wants to intervene, again the one with higher share
price has priority. If no corporation intervenes, the announcing
corporation \emph{must} now buy the company (i.\,e. no fake
announcements allowed).

\textbf{TODO: Perhaps put this paragraph in its own box with
  corporation's logo.} The special ability of the Orion corporation
affects the interaction with the foreign invester. Orion always has
first priority (as if its share price is higher than any other), and
Orion only pays \emph{face value} rather than the maximum allowed
price.

\subsection{Phase 4 -- Closing}

As in turn 1, it rarely makes sense to close companies so
early in the game. Keep ignoring this phase.

\subsection{Phase 5 -- Income}
\label{t2p5}

For players and the foreign investor, this phase works exactly the
same as in turn 1. The new thing is that corporations, too, collect
income.

Their base income is calculated in the same way as for players and the
foreign investor: Just add up all the income of the individual
subsidiary companies of a corporation. However, corporations (and only
corporations!) have the ability to generate bonus income from
\emph{synergies}:

A pair of companies that are subsidiaries of the same corporation and
have each other's abbreviation printed in one of their synergy boxes,
generates the bonus income printed in the upper left corner of the
synergy box. All pairs you can find within a corporation generate
bonus income, but each pair generates the bonus only once. Use synergy
markers to track the bonuses. For each pair, place a corresponding
synergy marker on the company card of the company with the higher face
value. Place it on top of the abbreviation of the other
company. (There is a bold red or yellow line in each synergy box. You
will see that the markers are only placed \emph{in front of} that
line, never behind. Never place markers behind the line (on the
abbreviations of companies with higher face value) to avoid
double-counting of bonuses.) To easily find all existing pairs, sort
the companies in descending face value order from left to right. Then
start with the left-most company and check all the companies listed in
its synergy boxes until you hit the bold line. No need to check behind
the line. Then repeat the procedure with the second left-most company
etc. You only ever have to look to the right. The companies left of
the one you are checking have already been checked. So you have to
look at fewer and fewer companies. The right-most company will never
receive a synergy token.

Certain corporations have special abilities to generate additional
bonus income.

\textbf{TODO: Perhaps put this paragraph in its own box with
  corporation's logo.} The Jupiter corporation receives +1\$ for each
company it owns.

\textbf{TODO: Perhaps put this paragraph in its own box with
  corporation's logo.} The Saturn corporation doubles the printed
income of its best company.

\textbf{TODO: Perhaps put this paragraph in its own box with
  corporation's logo.} The Horse corporation receives +1\$ for every
two synergy markers it owns (rounded down).

The bank pays the total income of a corporation (base income plus
synergy bonuses) into its treasury. For large corporations, it makes
sense to track the income on a sheet of paper so that you don't have
to calculate the total income again each turn.

\textbf{TODO: Add figure illustrating the example below.}

\emph{Example: The Horse corporation in the figure above consists of
  the DSB, the MS, the BPM, and the BSE. Its base income is \$5 + \$3
  + \$2 + \$1 = \$11. The DSB pairs with the MS, yielding +\$2. The MS
  pairs with both, the BPM and the BSE, for +\$1 each. Finally, the
  BPM pairs with the BSE for +\$1. The synergies add up to +\$5 (note
  the synergy markers). Finally, the special ability grants another
  +\$2 income for four synergy markers. Thus, the total income is \$11
  + \$5 + \$2 = \$18.}

\emph{You can see here how the synergies model a network: Both the BPM
  and the BSE historically started in Berlin. The MS is the state
  railroad of Mecklenburg, a German state north of and not far from
  Berlin. So it connects to the two Berliner railroad companies. The
  DSB, in turn, is the state railroad of Denmark, which is north of
  Germany, and relatively close to Mecklenburg. So MS and DSB can
  connect their networks for mutual benefit. The Horse corporation has
  developed into an international northern European railroad trust.}
\columnbreak


\subsection{Phase 6 -- Dividends}
\label{t2p6}

Starting with the corporation with the highest share price card, and
then continuing in descending share price order, each corporation pays
a dividend (which is chosen by the president and can be as low as \$0)
and then adjusts its share price.

For each corporation, the president performs the following steps:
\begin{enumerate}
\item The top card of the stack of shares on your symbol card (or the
  symbol card itself if all stacks have been issued) tells you how
  many shares you have issued. You will have to pay dividends to each
  of those shares. So keep that number in mind.
\item On your share price card, you see the maximum possible dividend
  per share. Obviously, the corporation's treasury must have enough
  money to pay the dividends. If you have three shares issued and \$8
  in the treasury, the maximum dividend per share is \$2, even if the
  share price card allows more. The minimum dividend is \$0 (you can
  call that ``not paying a dividend'', it doesn't make a
  difference). Pick a dividend in this range, and pay it from the
  corporation's treasury to the owner of each share (which might be
  yourself (this time as a player, not a president), another player,
  or the bank).
\item Now count the ``stars'' of the corporation: Add the
  number of stars you see on each company card owned by the
  corporation. Every full \$10 in cash adds another
  star. \textbf{TODO: Replace the word \emph{star} by the star symbol?
    If play money exists, mention the star symbol on the notes, and
    that smaller change still provides a star for each \$10 it ads up to.}
\item Take your share price card and look at the bottom half. Find the
  number of stars corresponding to the number of issued shares and
  compare it to the number of stars you have just counted to find your
  new share price card. If the two numbers are equal, nothing happens
  and you keep your old share price card. If you have counted one
  fewer star than printed on the share price card, your new share
  price card is ``one down'', i.e. it is the share price next to the
  single left arrow in the upper left corner of the share price
  card. Correspondingly, if you have counted one more star than
  printed on the share price card, your new share price card is ``one
  up'', i.e. it is the share price next to the single right arrow in
  the upper right corner of the share price card. If you have counted
  two \emph{or more} stars fewer or more than printed on the share
  price card, your new share price is ``double down'' or ``double
  up'', respectively, i.e. it is the share price next to the
  corresponding double arrow in the upper left or right corner,
  respectively. Return your old share price card to its spot in the
  row of share price cards and take your new share price card from the
  row. If your new share price card is currently in use by another
  corporation, find the next available share price card in the
  direction of the arrow.
\item Turn your new share price card vertically (to mark that you have
  gone through this whole procedure). Should you have kept your old
  share price card, still turn it vertically.
\end{enumerate}

\textbf{TODO: Add figure illustrating the example below.}

\emph{Example: Imagine the same corporation as in the previous example
  (section \ref{t2p5}). It has three shares issued, a current share
  price of \$22, and after paying dividends, it has \$21 left in
  treasury. Its ``star count'' is therefore 3 (for the DSB) + 2 (for
  the MS) + 1 (for the BPM) + 1 (for the BSE) + 2 ( for \$21 cash) =
  9. With three shares issued, the share price card asks for 7
  stars. Since you have counted two more, the new share price is \$27,
  as marked in the upper right corner next to the double arrow. If the
  \$27 share price card is not available, the next higher available
  share price card must be taken. Note that the location of the \$24
  share price card doesn't matter here. A ``double up'' doesn't mean
  to go up two available share price cards. Always find the target
  share price first, and then start to ``leapfrog'' up or down from
  there on. (If the corporation had only \$2 less cash, its star count
  would be 8, and the new share price would be only \$24. If the
  corporations had less than \$10 cash, its star count would be only
  7, and the share price would stay the same.)}

\textbf{TODO: Perhaps put this paragraph in its own box with
  corporation's logo.} The Star corporation alway adds two additional
stars to its star count.

\subsection{Phase 7 -- End card}

We are still not close enough to the end of the game to make anything
happen in this phase. Keep ignoring it.

\subsection{Phase 8 -- Issue shares}
\label{t2p8}

Finally, corporations have the opportunity to issue new shares. After
going public, this is the only phase where the stack of shares on the
symbol card is touched and more shares enter the market.

Again in decreasing share price order (starting with the corporation
with the highest share price card), the president of each corporation
decides if the corporation issues one share or no share. (A
corporation cannot issue more than one share in this phase.) In any
case, the share price card is turned horizontally to mark that the
corporation already had the opportunity to issue a share.

Issuing a share works very similarly to selling a share. Perform the
following steps for the corporation that issues a share:
\begin{enumerate}
\item Place the top-most share from the pile of unissued shares on the
  corporation's symbol card into the bank. (If there are no shares
  left, i.\,e. all shares have already been issued, you cannot
  issue more shares. Sorry.)
\item Return the corporation's share price card to its place in the
  row of share price cards and replace it with the next lower
  available share price card. (Usually, that is the next lower card as
  marked by the single arrow in the upper left corner of the share
  price card. \textbf{TODO: Adjust to actual card design.}  However,
  as before, you will skip missing share price cards.)
\item Now the bank pays the \emph{new} share price into the
  corporation's treasury.
\end{enumerate}

Later in the game, it is possible that the share price card the
corporation has to take is the \$0 one. In that case, the corporation
is declared bankrupt and removed from the game. Follow the
instructions on the \$0 share price card. Note that shares and symbol
cards are ``recycled'', i.\,e. they may later be used to form new
corporations. However, the ``recycled'' shares have nothing to do with
the old bankrupt corporation. The bankrupt corporation is gone for
good, without compensation.

\textbf{TODO: Perhaps put this paragraph in its own box with
  corporation's logo.} When the Eagle corporation issues a share, its
share price does not change. It simply receives the current share
price from the bank after placing a share into the bank pool.

\subsection{Phase 9 -- IPO}

In general, this phase works the same as in turn 1. However, with the
more valuable companies that have entered the game by now, you will
sooner or later go public with a company whose face value matches or
even exceeds the initial share price. In case the share price matches
the face value, you simply don't have to pay anything into the
corporation's treasury. The share you get has precisely the value of
the company you have given up, so you are all set. But what if the
share price is lower than the face value of the company going public?
Very simple: Take \emph{two} shares rather than one. Start with the
president's share, and then take an additional share. After that, pay
the difference between the \emph{doubled share price} and the face
value of the company into the treasury of the corporation as usual. In
this case, you also put two shares into the bank pool. Of course, the
bank now pays twice the share price into the corporation's
treasury. You always end up with half of the issued shares in your
possession and the the other half in the bank.

\emph{Example 1: You go public with the BY (face value \$12). You
  choose a share price of \$12. You receive one share (the president's
  share) and pay nothing. The bank gets another share and pays \$12
  into the treasury of the new corporation. The corporation ends up
  with two shares issued and \$12 in its treasury.}

\emph{Example 2: You go public with the BY (face value \$12). You
  choose a share price of \$11. You take the president's share, but
  its value is not sufficient to match the face value of the BD. So
  you take another share. Now you have two shares with a total value
  of \$22. You pay \$10 from your cash into the treasury of the new
  corporation. The bank gets two shares, too, and pays \$22 into the
  treasury. The corporation ends up with four shares issued and \$32
  in its treasury.}

\emph{Example 3: You go public with the CDG (face value \$60). You
  choose a share price of \$30 (a share price that is only allowed for
  green and blue companies, but fortunately the CDG is a blue
  company). You take two shares. Their value matches the face value of
  the CDG, so you don't have to pay anything. The bank gets two shares,
  too, and pays \$60. The corporation ends up with four shares issued
  and \$60 in its treasury.}

\section{The remaining turns}

We are almost there. You only have to learn a few more things, which
only become relevant later in the game.

\subsection{Phase 1 -- Investment}
\label{t3p1}

Nothing really changes here compared to previous turns. However, we
have to deal with a few special cases:
\begin{itemize}
\item After selling a share, the new share price might be \$0. In that
  case, the corporation is bankrupt and the same procedure is
  triggered as described in section \ref{t2p8}.
\item After buying a share, the new share price might be \$75. In
  that case, the game ends after the buy action has been completed
  (i.\,e. you still have to pay the \$75 for the share you have just
  bought, but after that, the game is over). Read on in section
  \ref{win} to learn how to determine the winner.
\item Eventually, the deck of companies will run out. The last card in
  the deck is the game end card. It is never drawn and just stays
  where it is. If you cannot draw a company after an
  auction, just skip that step. The offering will contain one fewer
  company whenever that happens.
\item Eventually, there will be no companies left in the
  offering. From that point on, the action \emph{start an auction}
  cannot be chosen any longer.
\end{itemize}

\subsection{Phase 2 -- Wrap-up}

This phase works exactly the same as in previous turns throughout the
game.

For convenience, you can stop tracking the money of the foreign
investor once there are no companies in the offering anymore.

\subsection{Phase 3 -- Acquisition}

This phase works exactly the same as in turn 2 throughout the game.

\subsection{Phase 4 -- Closing}
\label{t3p4}

As you will see in the next section, later in the game, a \emph{cost
  of ownership} will apply to certain companies. You might find
yourself (or your corporation) in a situation where you want to get
rid of one or more companies. In this phase, you can remove any number
of your privately owned companies from the game. Essentially, you can
do the same for the companies owned by corporations you
control. However, a corporation has to retain at least one subsidiary
company at any time.

If your total income from your privately owned companies in the
following phase 5 (see section \ref{t3p5}) will be negative and you
don't have enough money to pay for it, you \emph{must} close enough
companies in this phase to be able to pay for your losses (or get rid
of the losses altogether). In other words: As a player, you cannot
drive yourself into bankruptcy.

As in phase 3, the players act in no particular order. They simply
close companies as they see fit, and once nobody wishes to close a
company any longer, the phase ends.

The foreign investor automatically closes any companies with a
negative income (to be vetted separately for each company).

\textbf{TODO: Perhaps put this paragraph in its own box with
  corporation's logo.} Whenever the Ship corporation closes a company,
it immediately receives twice the printed income of that company as a
scrapping bonus.

\subsection{Phase 5 -- Income}
\label{t3p5}

The income calculation works the same as before, but at some point in
the game, you will have to deduct a \emph{cost of ownership}. Refer to
the back of the top-most card of the deck of (not yet drawn)
companies. Starting with the green cards, it will show a central
rectangle with a cost of ownership. Each company matching any of the
colors in the rectangle suffers the cost of ownership printed on the
card, i.\,e. its income is reduced, possibly becoming negative. Each
player and each corporation first add up the income of all their
companies, and only then they receive or pay the total income (if it
is positive or negative, respectively). A player will always be able
to pay their negative income (because they were required to close
companies to make sure of that, see section \ref{t3p4}). However, a
corporation might not be able to pay its negative income. In that
case, it goes bankrupt. Treat it the same as if it has just reached
share price \$0 (see section \ref{t2p8}).

Once all company cards have been drawn from the deck, the game end
card is visible. With regard to cost of ownership, it is treated the
same as the central rectangle on the company cards.

\textbf{TODO: Add figure illustrating the example below.}

\emph{Example: Look at the same corporation as used in the example in
  section \ref{t2p5}. We calculated an income of \$18. That was
  without cost of ownership yet. If the top face-down card of the deck
  is a green one (see figure above), each red company has a cost of
  ownership of \$2. Our corporation would earn \$2 less per red
  company it owns. So it would earn \$4 less, its income would be
  \$14. Once the top-most company of the deck is a blue one, each red
  and orange company earns \$4 less. The Horse would only earn \$18 --
  3*\$4 = \$6. Its president might be tempted to close the two red
  companies. Let's do the math: After that, the Horse would only own
  the DSB and the MS. That's \$5 + \$3 base income, \$2 for the
  synergy between the two, nothing anymore for the Horse's special
  ability (only one marker doesn't make a dent), and finally a
  deduction of \$4 cost of ownership for the orange MS. Again, we
  arrive at \$6 total income. Since closing the red companies reduces
  the star count of the Horse, it's usually a better deal to not close
  them yet.}

\textbf{TODO: Add figure game-end-card front.}

\emph{Once all companies have been drawn, the front-side of the game
  ending card is visible (see figure above). From now on, each yellow,
  orange, and red company has a cost of ownership of \$7. Our Horse
  corporation would earn 4*\$7 = \$28 less than the unmodified
  income. In total, it now makes \$10 losses. Even if it closed all
  but the best company (DSB), it would still make \$2 losses per
  turn.}

\textbf{TODO: Add figure game-end-card back.}

\emph{In the last turn of the game (see section \ref{t3p7} below),
  the game end card will be flipped, and the cost of ownership will
  not only be much higher, it will even affect green companies. (Luckily,
  it will only last one turn.)}

\textbf{TODO: Perhaps put this paragraph in its own box with
  corporation's logo.} The Bear corporation reduces its cost of
ownership by up to \$10 (but not below \$0).

\subsection{Phase 6 -- Dividends}

Once more, nothing really changes here compared to previous turns, but
eventually, you might run into one of the following special cases:
\begin{itemize}
\item After adjusting the share price, the new share price might be
  \$0. In that case, the corporation goes bankrupt and the same
  procedure is triggered as described in section \ref{t2p8}. (Note
  that paying dividends happens first. It's perfectly legal to pay a
  dividend only to drop to \$0 and go bankrupt right after that.)
\item After adjusting the share price, the new share price might be
  \$75. In that case, the game does \emph{not} end immediately. The
  phase continues, and only in the subsequent phase 7 the game will be
  declared over. Should it happen that other corporations reach \$75,
  too, those companies don't take a new share price card. They only
  return their old one. The shares of a corporation without a share
  price card have a share price of \$75.
\end{itemize}

\subsection{Phase 7 -- End card}
\label{t3p7}

Eventually, something will happen in this phase.

First you have to check if the \$75 share price card is held by a
company. If so, the game ends immediately. Read on in
section~\ref{win} to learn how to determine the winner.

If the \$75 share price card is not in use, check if there are still
companies available for auctions in the offering. If not, flip the
game end card (which will increase the cost of ownership). Once you
reach phase 7 again, the game ends. (In other words: If at the start
of phase 7, the game end card is already flipped, the game ends in the
same way as if the \$75 share price card were in use.) Again,
section~\ref{win} tells you how to determine the winner.

While the game end card is flipped (i.\,e. during the last turn of the
game), the game might still end in phase 1 as described in section
\ref{t3p1}.

\subsection{Phase 8 -- Issue shares}

This phase works exactly the same as in previous turns throughout the
game.

\subsection{Phase 9 -- IPO}

This phase works exactly the same as in previous turns throughout the
game.

\section{Receivership}
\label{rec}

\textbf{TODO: Put this into a box rather than a section?}

For as long as all issued shares of a corporation are owned by the
bank, that corporation is \emph{in receivership}. As a reminder, place
one of the red receivership cards next to the corporation's
assets. The card also lists the rules that now apply to the
corporation. Essentially, the corporation is now run by a receiver
(who is not always acting very smart). The rules in a bit more detail
than on the receivership card:
\begin{description}
\item[Phase 1:] The first player that buys a share of an corporation
  in receivership must pick the golden president's share and
  immediately becomes the new president, thereby ending receivership.
\item[Phase 3:] Corporations in receivership are very eager to buy
  from the foreign investor. However, in most cases, they don't have
  enough cash. In practical terms, you should check at the beginning
  of phase 3 if any corporation in receivership is able to buy a
  company from the foreign investor at all. If that's the case, start
  with the most expensive eligible company and pretend that the
  receiver has just announced the intention to buy the company. Then
  follow the usual rules of intervention and priorities (including the
  special ability of the Orion corporation). Repeat with the less
  expensive companies until no companies are eligible anymore.
\item[Phase 4:] Similar to the foreign investor, the receiver tries to
  close companies that make losses. However, since synergies might be
  generated by corporations, even in receivership, only close red
  companies if the cost of ownership is at least \$4 per company, and
  only close orange companies if the cost of ownership is at least \$7
  per company. Never close yellow, green, or blue companies. If the
  corporation needed to close \emph{all} its companies, keep the one
  with the highest face value. (Note that this rule won't always do
  the ``right'' thing. Companies might be closet that are still
  profitable, and unprofitable companies might stay open. As said, the
  receiver is not very smart.)
\item[Phase 5:] A corporation in receivership may go bankrupt as
  normal if it cannot pay for its negative income.
\item[Phase 6:] The dividend of a corporation in receivership is
  always \$0.
\item[Phase 8:] A corporation in receivership will always issue a
  share if at all possible, even if it goes bankrupt by doing so.
\end{description}

\section{How the game ends}
\label{end}

As described above, there are three ways the game may end:
\begin{itemize}
\item If a corporation takes the \$75 share price card during a
  \emph{buy one share} action in phase 1, the game ends immediately
  after that action is completed.
\item If the \$75 share price card is in use during phase 7, the
  game ends.
\item If phase 7 starts with the game end card already flipped, the
  game ends.
\end{itemize}

\section{Who has won?}
\label{win}

For the final ranking of players, add the value of everything each
player owns:
\begin{itemize}
\item Their cash.
\item The face value of each private company they own.
\item The current share price of each share they own.
\end{itemize}

For the final player ranking, it is irrelevant how much cash and which
companies the corporations own.

If there is a tie, break it by player order (lower number in player
order wins over higher number).

\section{Easily missed or misunderstood rules}

The following list of things beginners often get wrong might prove
helpful in your first couple of games. It is in approximate order of
frequency, most common issues first.

\begin{itemize}
\item Synergies are only possible within a corporation. Companies
  owned directly by a player or by the foreign investor never ever
  receive synergy bonuses. When counting synergies, count every pair
  only once. If A synergizes with B, then B will always synergize with
  A, too. You still get the bonus only once.
\item \emph{Pass} and \emph{leaving an auction} both happen during
  phase 1, but are entirely different things. \emph{Pass} is an action
  you may take when it's your turn to perform one action. If you do
  that, you basically do nothing. If all players pass
  consecutively, phase 1 is over. But if any of the others take a
  non-pass action, you will have another turn, and when it's your turn
  again, you may (and must) choose a new action (which might be
  \emph{pass} again, but any other legal action is eligible, too). In
  other words: Passing doesn't prevent you from taking another action
  later. In contrast, if you \emph{leave an auction}, you have left
  the auction for good. You may not bid in the same auction ever
  again. Strictly speaking, \emph{leaving an auction} is not an action
  at all. It happens as a sub-step during an auction, which is
  triggered by any player's \emph{start an auction} action.
\item After an auction, keep in mind that the last player that has
  performed an action is the player that has \emph{started} the
  auction (\emph{not} the player that has won the auction). So the
  next player performing an action will be the one next in player
  order to the player that has started the auction.
\item Never transfer any assets (money, shares, companies) in a way
  not explicitly allowed by the rules. You can't sponsor your
  corporations, you can't ``steal'' from the treasury of your
  corporations, you can't give money or companies to other players,
  not even as a gift, etc. Keep all assets next to their respective
  owner (players, corporations, foreign investor, bank) and clearly
  separated from others.
\item It is very tempting to think of the share price cards you see on
  the table as the price you have to pay to buy a share (or the price
  the bank will pay you if you sell a share). However, you have to pay
  the next higher available share price (and you will be paid the next
  lower available share price). You can see the next regular share
  prices in the corners of the share price cards next to the single
  arrows, but remember that cards that are already in use are skipped,
  so the relevant price may be even higher (or lower, in case you
  sell). \textbf{TODO: Adjust to actual share price card design.}
\item Newly drawn companies are not available for auction in the same
  turn. They have to wait until next turn. (Even the foreign investor
  cannot buy them in phase 5 of the same turn.)
\item Never ever use any \$ or any company twice in phase 6. Don't
  forget to turn vertically the companies and the money used. Execute
  each transaction separately. Things like ``The Bear buys MHE for \$8
  from the Eagle, and at the same time the Eagle buys the BPM from the
  Bear for \$8, too, so we just swap companies and no money'' don't
  work. First transfer one of the companies (let's say the MHE) and
  pay the money (and turn both vertically), then do the same with the
  other company (the BPM), pay the money (which must not be the money
  turned vertically), and turn them both vertically.
\item The cost of ownership is defined solely by the back of the
  top-most card in the deck of unrevealed company cards (or, if the
  deck has run out, by the game end card left behind). Once a company
  card has been drawn, it will never be flipped back again and its
  back is irrelevant for the rest of the game.
\end{itemize}
\end{multicols}

\chapter{Notes for players of the original
  \emph{Rolling Stock} or 18xx games}

\begin{multicols}{2}

\textbf{TODO: Vet if we want this chapter in the printed player's
  guide at all.}
  
This chapter is for those that already have previous knowledge by
having played the original \emph{Rolling Stock} game and/or 18xx
games. This knowledge can be extremely helpful when grasping
\emph{Rolling Stock Stars}, but there are also some caveats where you
have to ``un-learn'' certain aspects of those games.

\section{If you have played the original \emph{Rolling Stock}\dots}

\dots you are probably asking yourself now why you should play
\emph{Rolling Stock Stars} instead of (or even in addition to) the
original game. There are two obvious, but radically different
conclusions people might jump to: The one extreme is the notion of
\emph{Rolling Stock Stars} as an ``improved'' version of \emph{Rolling
  Stock}: After years of experience with the original game, it was
certainly time for a 2nd, more polished edition which quirks
removed. The other extreme is the verdict of \emph{Rolling Stock
  Stars} being a ``streamlined'', ``more-forgiving'', or even
``watered-down'' version of the original game. The former would make
you prefer \emph{Stars} over the original game in any case, while the
latter would position \emph{Stars} as a learning game, over which
experienced players would prefer the original game.

In reality, it's not that easy. Both conclusions are not entirely
wrong, but still more wrong than right. \emph{Stars} would certainly
not have been possible without the experience many players made with
the original game. Also, \emph{Stars} is indeed meant to be more
approachable and less niche than the original game. Still, it would be
wrong to generally think about one game as fundamentally ``better''
than the other, or as \emph{Stars} being only meant for beginners or
more casual players. Both games are much more different than the
seemingly small differences would suggest. While there are experienced
players preferring the original game over \emph{Stars} (or vice
versa!), some just claim that the games are too different to be ranked
directly against each other.

\textbf{TODO: Finish!}

\section{If you are an 18xx player\dots}

\dots many concepts in \emph{Rolling Stock Stars} will be familiar to
you. However, there are a number of significant differences. The
following list will help you to avoid the most common traps for 18xx
players.

\begin{itemize}
\item Players start with \$30 for every player count except
  6. (Basically, instead of decreasing the money of each player, the
  game size is increased to accommodate more players.)
\item Pay special attention to phases 3 and 4, which are performed in
  ``any order''. Don't wait until it's ``your turn''. Just act.
\item In phase 1 (you might want to call it ``share round''), you
  indeed have exactly and only one action whenever it is your
  turn. \emph{Either} buy \emph{or} sell \emph{or} start an auction
  \emph{or} pass. And if you sell, it's only ever \emph{one} share per
  action.
\item Otherwise, share trading has almost no restrictions compared to
  18xx. There is no certificate limit. There is no limit of shares in
  the pool. There is no limit of shares an individual player may hold
  (may be 100\,\%). You may sell shares of a corporation that has just
  been founded. You may even buy shares you have sold before in the
  same phase. (Oh yes! But keep in mind the next item below. In other
  words: If you keep selling and buying the same share, you will lose
  money each time.)
\item Every individual sell and buy action will modify the share price,
  and you will get/pay the \emph{new} share price (see also the
  non-18xx-specific notes above).
\item At the end of phase 1 (the ``share round''), fully sold shares
  will \emph{not} change their share price.
\item There is no notion of a share being explicitly a ``10\,\%
  share'' or a ``20\,\% share''. Keep in mind that shares not yet
  issued basically don't exist. (After the IPO, the only way
  un-issued shares enter the game is by issuing shares in phase 8.)
  If a company has two shares issued, each is implicitly a 50\,\%
  share. If it has three shares issued, each is 33\,\%, and so
  on. Also note that the president's share is a single share, not a
  double share.
\item There is no ``emergency money raising''. If your corporation
  has a negative net income and cannot pay for it, it goes bankrupt.
\item You set a dividend per share and then pay it from treasury. The
  dividend you pay has no direct link to the income of your
  corporation in the same turn. Even if your corporation has a
  negative income, it may still pay dividends (if there is enough
  money left in the treasury). Furthermore, the share price adjustment is
  not directly coupled to the dividends you pay (despite this happening
  in the same phase 6). It is indirectly coupled (via the star count),
  but the effects are the opposite of what you would expect: In
  general, paying a dividend makes it more likely your share
  price will drop, while not paying a dividend (strictly speaking:
  paying a dividend of \$0) makes it more likely your share price
  will increase.
\item In a certain way, the companies in \emph{Rolling Stock Stars}
  are a bit of both, privates and trains in 18xx. However, there is no
  upper limit of the number of subsidiary companies in a corporation
  (no ``train limit''), and companies are never scrapped by
  force. (The latter is, however, not entirely alien to the 18xx
  world. \emph{1873 Harzbahn} uses a very similar cost-of-ownership
  system.)
\item Phase 4 (new player order) works exactly like in \emph{1844:
    Switzerland}. If you know that game, nothing new here. Otherwise:
  It's basically a refined priority deal.
\item The bank has unlimited money.
\end{itemize}


\end{multicols}


\chapter{Variants}
\begin{multicols}{2}

The following introduces several possible variants.

\textbf{TODO: Vet for each if we want it in the printed player's guide.}

\section{Deals and negotiation}
\label{deals}

\textbf{TODO: Concern here is that this is mostly of general concern
  for many games of a similar type. It might not be worth it to put
  this into the printed player's guide where people would expect more
  specific information.}

The rules are (intentionally) silent about deals and
negotiations. Rules about deals and negotiations are a bit like rules
about showing up on time to start the game or switching off your
mobile phone while playing. Things are different for games with secret
information, i.\,e. where some players have information others
don't. In that case, you need rules about legal ways to share (or not
to share) this information. But \emph{Rolling Stock Stars} has no secret
information. Of course, the order and composition of the deck has a
random component, but no player knows more than any other.

So by default, players can just say whatever they want. Nothing is
forbidden, but nothing is enforced either. Feel free to forge deals
and alliances, but remember that the rules won't help you to enforce
those deals. (I believe it is basically impossible to write consistent
rules that would make freeform deals binding. Deals are too often
worded ambiguously, or they can't be fulfilled without breaking the
rules, or a player has agreed to multiple deals that are mutually
exclusive.) There is little danger that \emph{Rolling Stock} would
degenerate into a \emph{Diplomacy}-style backstabbing game, simply
because long-term deals are rare and the short-term deals neither
require nor foster a long-term partnership (if at all, those will
implicitly emerge from overarching strategic goals, e.\.g. a single
player is running away with the game so that the other players
cooperate with each other more intensely to catch up -- perhaps they
will even manage to implement an embargo against the leading
player). In JC Lawrence's words: Both sellers and buyers (in phase 3)
are ``naturally promiscuous''.

Groups might have their own etiquette about deals and
negotiations. Feel free to implement whatever you feel is right.
However, I'd strongly discourage from secret negotiations. They would
be a huge time drain, and I believe they are neither in the spirit of
the game nor will players feel a great need for them.

In general, you should make sure that negotiations don't stall the
game for too long. If you can't avoid spending
an uncomfortable amount of time with negotiations and/or if you want
to limit negotiations for other reasons, try one of the following more
formalized variants:

\begin{itemize}
\item Strictly limit the time for the ``any order'' phases (e.\,g. two
  minutes for phase 3 and one minute for phase 4, feel free to use any
  value you see fit). In all ``sequential'' phases, players have to
  decide quickly and must not negotiate with other players when it's
  their turn to do something. At any other time, they may negotiate
  freely.
\item Strictly limit negotiations to phase 3. (The more experienced
  players become, the more they will feel the need to plan in
  advance. The decision to issue a share in phase 8 or to form a
  corporation in phase 9 depends on future deals in phase 3 of the
  next turn. Players might be tempted to meticulously arrange all
  those deals for phase 3 as soon as in phase 8 or 9 of the previous
  turn, which might stall the game quite seriously.)
\item The most radical solution is a strict ``no deals, no
  negotiations'' policy. In phase 3, offers and counter-offers can
  still be made, but without additional table-talk. The following will
  still be OK: ``Do you want to buy the MAD for \$50?'' -- ``I'll give
  you \$45.'' -- ``Let's say \$47.'' -- ``Deal.'' Not OK would be any
  additional arguments along the lines: ``I can't give you more than
  \$45 because I still need these \$12 left to buy the PR from Chris's
  Horse corp. Furthermore, the \$45 are good enough for you because
  that will allow you to pay dividends and still rise in share
  price.'' This radical variant is most suitable for ``blitz''
  games. You might manage a full game in only 90 minutes. But keep in
  mind that ``Rolling Stock Stars'' is a very interactive game, and
  negotiations and deals are supposed to be part of the fun.
\end{itemize}

There is one specific type of situation where a certain type of
players might create a sense of backstabbing.

\emph{Example: Alice is the president of the ``Android'' corporation,
  which owns the WT and the OL. Bob is the president of the ``Bear''
  corporation, which owns the MS and the BY. Alice and Bob agree that
  they should ``swap'' the OL and the BY to get better
  synergies. Since a direct swap is not possible, what formally has to
  happen is two transactions: (1) The ``Bear'' buys the OL from the
  ``Android''. (2) The ``Android'' buys the BY from the
  ``Bear''. Alice and Bob agree to do both transactions for the
  minimum possible price of \$7 (because both corporations are short
  of cash at the moment). The order of the transactions doesn't really
  matter, but you have to start somewhere. So Alice's ``Android''
  hands over the OL to Bob's ``Bear'', and the ``Bear'' transfers \$7
  to the ``Android''. Alice wants to go on and to execute the second
  transaction, but in that moment, ``all of a sudden'', Bob has second
  thoughts and refuses. Alice feels backstabbed. Bob's behavior is
  completely legal, though. The rules don't enforce any connection
  between transactions.}

If this kind of situation appears to be a problem in your games, you
might want to introduce a variant rule that allows ``complex''
transactions where a number of individual transactions can be executed in
one step (so that the kind of ``second thoughts'' Bob had in the
example are rendered impossible). But make sure that the ``complex''
transaction would still be legal if executed in a series of individual
transactions. It is still impossible to ``swap'' companies if the
corporations don't have enough money to pay for their newly acquired
companies, or if both corporations only own one company.

\section{Secret private money}

\textbf{TODO: Concern here is what's written below anyway: If you
  don't forbid note taking, money is anyway trackable, even by players
  with weak memory.}

In the rules, all assets are open for inspection. Some players,
however, prefer to play with secret private money. (The treasury of
corporations has to be open because the stars have to be counted in
phase 6.) Feel free to do so as a variant, but keep in mind that the
private money is perfectly trackable. If you allow players to take
notes on paper (which is strongly encouraged to speed up the game),
then tracking the private money of each player becomes merely a matter
of diligence, and most players will probably argue you should simply
play with open money to spare everybody the tedious tracking work. If
you disallow notes (or only allow specific kind of notes), tracking
private money becomes a brain exercise, which only some players
consider fun. Others simply won't bother and leave it to their
intuition, which will make auctions less predictable (``How much money
will he have?  How much do I have to bid to kick him out of the
auction?''). Again, some players will consider that fun, others
not. It's your call.

\section{Open companies}

Some players dislike the unpredictability of the deck. To solve that,
you can play with an open deck. Build the deck as usual, but then
declare it open for inspection. To facilitate inspection, you can turn
all company cards face-up. In that case, you should use one each of
the unused green and blue company cards to mark the current cost of
ownership. (Once the top-most card of the deck is green, place the
unused green company card face-down next to the deck. Correspondingly,
do the same once the top-most card is blue.)

\section{Two-player variant}

\textbf{TODO: I would either toss this out completely or make it
  ``official'' (so that we can label the box as \emph{2 to 6
    players}. I guess this needs more test games for the final call.}

To play with two players, set up the game as if you were playing with
four players. Then, each player takes the position of two players in
the four-player game simultaneously. Player A starts with his
simulated ``right hand'' player in player order position 1. Player B
starts with his simulated ``left hand'' player in position 2 and his
simulated ``right hand'' player in position 3. Finally, player A
starts with his simulated ``left hand'' player in position 4.

Play the game normally, as if it were a four-player game. To win, your
\emph{lower ranked} simulated player must be better than the lower
ranked simulated player of your opponent.

\emph{Example: Alice's ``right hand'' player ends up first in the
  final ranking with a huge margin, and her ``left hand'' player ends
  up on a close last place. Bob's simulated players end up on rank 2
  and 3, very close to Alice's ``left hand'' player. Bob wins the game
  because his lower ranked player is better than Alice's lower ranked
  player.}

\emph{Rolling Stock} is full of win-win deals. Forging those deal
between opponents isn't really interesting any longer in a two-player
game, because there is no third (or forth or fifth) party any longer
relative to which the two dealing players would win. While deals
between opposing players are in theory still possible in the
two-player variant, they would only happen if the two players had a
different understanding of the benefits of the deal and were both
thinking they were winning more than the other. Deals between
``allied'' simulated players are obviously highly encouraged, and the
two-player variant is very suited as an exercise for cooperative
strategies. You even have to make sure that both simulated players
benefit in a similar way because you can only win if you balance the
result of your two simulated players. That's very similar to a real
four-player game. (Of course, you can ``switch camps'' at any time in
a real four-player game and forge deals ``promiscuously'' with
changing partners, while the two camps are fixed in the two-player
variant.)
\end{multicols}

\chapter{Overview of companies}

\textbf{TODO: I think we should have the list of companies for
  reference. Can be rather short. Question is if we also want the
  ``flavor texts'', which I have left in below to vet.}

As an homage to the 18xx series of railroad games, games from that
series that feature one or more of the companies represented in
\emph{Rolling Stock Stars} are mentioned here.

\section{Red companies}

The red companies are early Prussian railroad companies from the first
half of the 19\textsuperscript{th} century.  The same six companies
are represented as \emph{Vorpreußen} in Michael Meier-Bachl's
\emph{1835} (with slightly different names, though). Some of the
companies can also be found in other games: The MHE in Klaus
Kiermeier's \emph{1873 Harzbahn}, the BPM and the BSE in David Hecht's
\emph{18EU}, the BME and KME in Wolfram Janich's \emph{18Rhl --
  Rhineland}, and the AKE in Wolfram Janich's \emph{1842: Schleswig
  Holstein}.

\section{Orange companies}

The orange companies are the railroads of the various German states in
the middle of the 19\textsuperscript{th} century.  Again, you will
find the same companies (with slightly different names) in Michael
Meier-Bachl's \emph{1835}. In addition, David Hecht's \emph{18EU}
features the BY and the PR, and Wolfram Jahnich's \emph{18SX} the SX.
In \emph{Rolling Stock Stars}, these companies start as private companies,
in the 18xx games, they are corporations.  Thus, the games 
somewhat misrepresent history, as all these companies were
state-owned.

\section{Yellow companies}

The yellow tier of companies covers the late 19\textsuperscript{th}
and early 20\textsuperscript{th} century. The DR is the state railroad
of the now unified German Empire, while all the other yellow companies
represent the railroad companies of the countries neighboring
Germany. Again, these companies were mostly state-owned. Representing
them as tradeable companies is once more bending history a bit. You
can find many of these companies in David Hecht's games: The SNCF, B,
DR, NS, and KK in \emph{18EU}, the SNCF and B also in \emph{1826}, and
the DSB in \emph{18Scan}. Leonhard Orgler's \emph{1837} features the
KK, as does Leonhard Orgler's and Helmut Ohley's \emph{1824}. The SBB
is the largest company in Peter Minder's and Helmut Ohley's
\emph{1844: Switzerland}.

\section{Green companies}

Historically, we are now moving deep into the 20\textsuperscript{th}
century. Geographically, we are expanding towards the periphery of
Europe. Two companies are not strictly railroad companies: The E
(representing the tunnel between Britain and France) and the BSR (a
hypothetical company running the ferries, bridges and tunnels in the
Baltic Sea). David Hecht's games feature two of the green companies:
The FS in \emph{18EU} and the SJ in \emph{18Scan}.

\section{Blue companies}

The blue tier of companies contains no railroad companies at all, but
the modern seaborne and airborne competitors. HA, HH, and HR are the
three largest container ports in Europe. MAD, LHR, CDG, and FRA are
the four largest European airports. Passengers and cargo have to reach
the ports and airports, so those companies are not only competitors of
the railroad companies but also offer some opportunities to
synergize.

\textbf{TODO: I have elided the credits. With the development of the
  game as it happened, I cannot really name ``core'' playtesters
  anymore, and listing them all (as in the original game) has become
  plainly impossible. Depending on tast, we could add a disclaimer
  like this, and then just have a few special persons mentioned: Scott
  as the publisher/developer, Toby for the online version etc. But I feel
  it's probably better to skip the credits entirely and perhaps
  mention the online version in its own section somewhere in the beginning.}

\end{document}

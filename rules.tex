\documentclass[11pt,a4paper]{article}
\usepackage[utf8]{inputenc}
\usepackage{multicol,anysize}
\usepackage{sfmath,mathabx,wasysym}

%\renewcommand{\familydefault}{\sfdefault}

%\papersize{21.0cm}{29.7cm} 
\marginsize{1.0cm}{1.0cm}{1.0cm}{1.0cm}
\setlength{\parindent}{0pt}
\setlength{\parskip}{0.3ex plus 0.5ex minus 0.2ex}

\newcounter{itemcounter}
\newenvironment{my_enumerate}
{\begin{list}{\arabic{itemcounter}.}
  {\usecounter{itemcounter}\leftmargin=1.8em}
  \setlength{\itemsep}{1pt}
  \setlength{\parskip}{0pt}
  \setlength{\parsep}{0pt}
}
{\end{list}}

\newenvironment{my_description}
  {\begin{list}{}{\setlength{\labelwidth}{0pt}
   \leftmargin=1.2em
   \setlength{\itemindent}{-\leftmargin}
   %\setlength{\listparindent}{\parindent}
   \renewcommand{\makelabel}{\descriptionlabel}}
  \setlength{\itemsep}{1pt}
  \setlength{\parskip}{0pt}
  \setlength{\parsep}{0pt}
  }
  {\end{list}}

\newenvironment{my_itemize}
{\begin{list}{\labelitemi}{\leftmargin=1.2em}
  \setlength{\itemsep}{1pt}
  \setlength{\parskip}{0pt}
  \setlength{\parsep}{0pt}}
{\end{list}}


\begin{document}

\begin{center}
\section*{Rolling Stock}

A card game for three to six players. Rules v2.2. Game design by
Björn Rabenstein.
\end{center}

These are the complete and canonical rules. However, they are meant as
a reference and the ultimate ``source of truth'' and are written in an
extremely condensed and formal way. They should \emph{not} be used to
learn the rules. Read the chapter \emph{Learning the Game} in the soon
to be published \emph{Player's Guide} instead.

\begin{multicols}{2}

{
\small

  Terms printed \textsc{in small caps} are explained in the
  \emph{Glossary} or in the \emph{Procedures} section at the end of
  the rules.

\subsection*{Components}

The game contains 131 cards, 6 turn summaries, 2 booklets (the
\emph{Rules} and the \emph{Player's Guide}), 109 round synergy
markers, and an unlimited amount of \textsc{money}. (In practice, the
amount of \textsc{money} provided with the game is obviously
limited. Use other means of tracking money in the unlikely case it
runs out.) The 131 cards in more detail:
\begin{my_itemize}
\item 8 symbol cards, each featuring a different \textsc{corporation}
  symbol (\emph{Jupiter}, \emph{Saturn}, \emph{Bear}, \emph{Horse},
  \emph{Eagle}, \emph{Orion}, \emph{Ship}, \emph{Star}).
\item 44 \textsc{shares} for the 8 \textsc{corporations}, identified
  by their symbols (4 \emph{Bear}, 4 \emph{Star}, 5 \emph{Jupiter}, 5
  \emph{Saturn}, 6 \emph{Eagle}, 6 \emph{Orion}, 7 \emph{Ship}, 7
  \emph{Horse}).
\item 6 grey player order cards, numbered from 1 to 6.
\item 27 white share price cards showing \textsc{share prices} from
  \$0 to \$75, prividing the necessary information to \textsc{adjust
    share prices}. The share price cards from \$10 to \$37 have a
  colored \emph{IPO} rectangle to mark \textsc{share prices} that can
  be used to \textsc{form a corporation}.
\item 36 \textsc{companies}, 6 red ($\bullet$), 8 orange
  ($\blacktriangleup$), 8 yellow ($\sqbullet$), 7 green ($\pentagon$),
  7 blue ($\hexagon$). (Each geometric shape is used to mark the
  respective color on all game components to assist color-blind
  players.) The number in the upper right corner is the income. The
  number in the upper left corner is the \textsc{face value}, followed
  by an allowed price span (in parentheses) used for \textsc{company}
  trading in phase 3. Below the face value, a number of stars is
  printed, which are used during \textsc{adjusting share prices}. The
  \textsc{face value} is unique for each \textsc{company}. Other
  unique identifiers are a one to five letter code and the full
  \textsc{company} name, to be found in the center of the top half of
  the card. The bottom half of each card shows blocks of
  \textsc{synergies} with other \textsc{companies}.
\item 1 game end card.
\item 1 grey \textsc{foreign investor} card.
\item 8 red \textsc{receivership} cards.
\end{my_itemize}

The normal orientation of cards and money is called
\emph{horizontal}. In many situations during the game, cards or money
are turned \emph{vertically} to mark a special state of that
particular component.


\subsection*{Setup}

Each \textsc{player} receives \$30 from the \textsc{bank}, a turn
summary, and a random player order card, thus determining the initial
\textsc{player order}. In a 6-player game, each player recives only
\$25. Set aside the \textsc{shares} in 8 separate stacks, one for each
\textsc{corporation}, sorted with the 1\textsuperscript{st} share on
top. Add the corresponding symbol card to the bottom of each
stack. Those stacks are available to \textsc{form corporations} in
phase 9. Lay out the share price cards in a \textsc{row}, in ascending
order from left to right. Place the \textsc{foreign investor} card on
the table. The \textsc{foreign investor} receives \$4 from the
\textsc{bank}. Build the \textsc{deck} of \textsc{companies} in the
following way: Place the game end card at the bottom of the deck, the
lower cost-of-ownership side up. Find the \textsc{company} with the
highest \textsc{face value} for each color. (The names and \emph{face
  values} of those \textsc{companies} are printed red or orange
instead of black for easier identification.) Start one face-down stack
per color with those \textsc{compannies}. Shuffle the remaining
\textsc{companies} separately by color and, without revealing them,
add a number of \textsc{companies} to each stack that corresponds to
the number of \textsc{players}, with the following exceptions: In a
4-player game, add 5 orange \textsc{companies}; in a 5-player game,
add 7 orange \textsc{companies}; in a 6-player game, add all companies
of all colors. Remove all remaining \textsc{companies} from the game
(without revealing them). Shuffle each stack. Add the blue
\textsc{companies} to the deck on top of the game end card. Then add
the green \textsc{companies} to the top of the deck, then the yellow
\textsc{companies}, then the orange \textsc{companies}, and finally
the red \textsc{companies}. Place the deck in the middle of the
table. Draw and reveal a number of \textsc{companies} equal to the
number of \textsc{players}. The drawn \textsc{companies} are available
for future \textsc{auctions}. Start the first turn of the game.

\subsection*{Game Sequence}

The game is played in a series of turns. Each turn is divided into the
following phases (in that order):

\begin{my_enumerate}
\item In \textsc{player order}, players choose \emph{one} of the
  following actions: \textsc{buy one share}, \textsc{sell one share},
  start an \textsc{auction}, or pass. After a player has passed, their
  player order card is turned vertically. After each non-pass action,
  it is turned horizontally. As soon as all player order cards are
  turned vertically, the phase ends.
\item Determine the new \textsc{player order} by descending remaining
  \textsc{money}. Ties are broken by the old \textsc{player
    order}. Turn all player order cards horizontally and redistribute
  them between players to indicate the new \textsc{player order}.  In
  \emph{ascending} \textsc{face value order}, the \textsc{foreign
    investor} buys as many of the remaining \emph{available}
  \textsc{companies} as possible, paying \textsc{face value}. For each
  \textsc{company} bought in that way, draw a new one from the
  \textsc{deck} and turn it vertically. Once the \textsc{foreign
    investor} cannot buy any more \textsc{companies}, all
  \emph{unavailable} \textsc{companies} become \emph{available} for
  future \textsc{auctions}. (Turn their cards horizontally.)
\item In \textsc{any order}, \textsc{corporations} buy
  \textsc{companies} from other \textsc{corporations},
  \textsc{players}, or the \textsc{foreign investor}. Each transaction
  is executed separately and involves exactly one \textsc{company},
  which is transfered from the seller to the buying
  \textsc{corporation}, and the \textsc{money} payed, which is
  transfered from the buying \textsc{corporation} to the seller. A
  transaction only takes place if both \textsc{entities} involved
  agree on a price within the allowed price span printed on the
  company card. The transfered \textsc{company} and \textsc{money} are
  turned vertically until the end of this phase and cannot be used for
  further transactions. (Turn them back horizontally after the phase
  has ended. A \textsc{company} turned vertically may be the only
  \textsc{company} a \textsc{corporation} owns.) The \textsc{foreign
    investor} only sells at the maximum allowed price, and the
  intention to buy a specific \textsc{company} from the
  \textsc{foreign investor} must be announced before executing the
  transaction. Any \textsc{corporation} with a higher \textsc{share
    price} and enough \textsc{money} available may then immediately
  intervene and buy the desired \textsc{company} itself. If more than
  one eligible \textsc{corporation} wants to intervene, the one with
  the highest \textsc{share price} has priority. If there is no
  intervention, the announcing \textsc{corporation} \emph{must} buy
  the specified \textsc{company}. \textsc{Corporations} in
  \textsc{receivership} never sell companies and only buy from the
  \textsc{foreign investor}, which is handled right in the beginning
  of the phase in the following way: The \textsc{corporation} in
  \textsc{receivership} with the highest \textsc{share price} tries to
  buy the most expensive \textsc{company} it can
  afford. \textsc{Corporations} not in \textsc{receivership} may
  intervene as usual. Repeat until the \textsc{corporation} cannot
  afford to buy any more \textsc{companies} from the \textsc{foreign
    investor}. Then repeat the whole procedure for the
  \textsc{corporation} in \textsc{receivership} that has the next
  highest \textsc{share price} until all \textsc{corporations} in
  \textsc{receivership} are taken care of. During the whole phase, the
  \emph{Orion} \textsc{corporation} is considered to have the highest
  \textsc{share price} and pays only \textsc{face value} for
  \textsc{companies} bought from the \textsc{foreign investor}.
\item In \textsc{any order}, \textsc{players} and
  \textsc{corporations} may close \textsc{companies} by removing them
  from the game. \textsc{Players} must close a sufficient number of
  \textsc{companies} with negative income to be able to pay for a
  possibly negative total income in phase 5. The \textsc{foreign
    investor} closes \textsc{companies} whose \textsc{cost of
    ownership} exceeds their income. The \emph{Ship}
  \textsc{corporation} receives twice the printed income for each
  \textsc{company} it closes. \textsc{Corporations} in
  \textsc{receivership} close red \textsc{companies} if the
  \textsc{cost of ownership} is at least \$4 and close orange
  \textsc{companies} if the \textsc{cost of ownership} is at least \$7
  but always keep the \textsc{company} with the highest \textsc{face
    value}.
\item \textsc{Players}, \textsc{corporations}, and the \textsc{foreign
    investor} \textsc{collect income}.
\item In \textsc{share price order}, \textsc{corporations} \textsc{pay
    dividends}, \textsc{adjust their share price}, and turn their
  share price card vertically. \textsc{Corporations} in
  \textsc{receivership} pay a dividend of \$0.
\item If there are no unowned \textsc{companies} left, flip the game
  end card. If it is already flipped, or if the \$75 share price card
  is owned by a \textsc{corporation}, the game ends.
\item In \textsc{share price order}, each \textsc{corporation} may
  \textsc{issue one share}. Then it turns its share price card
  horizontally (whether it has issued a share or
  not). \textsc{Corporations} in \textsc{receivership} issue a share
  if they have any left.
\item In descending \textsc{face value order}, private companies may
  \textsc{form corporations}.
\end{my_enumerate}

Repeat the turn sequence until the game ends in phase 1 or phase
7. Then rank players according to their \textsc{book value},
breaking ties by \textsc{player order}.

\subsection*{Glossary}

\begin{my_description}
\item[Any order] During a phase executed in any order, players act (for
  themselves or as \textsc{presidents} on behalf of
  \textsc{corporations}) whenever and as often as they wish, even
  concurrently or alternating with other players. They may wait for
  other players' actions but when nobody has acted for a reasonable
  amount of time, the phase ends.
\item[Bank] An \textsc{entity} whose actions are entirely determined
  by the rules. Owns an unlimited amount of \textsc{money} and all
  \emph{issued} \textsc{shares} that are not owned by
  \textsc{players}.
\item[Book value] The sum of the \textsc{face value} of each
  \textsc{company} and the \textsc{money} owned by an
  \textsc{entity}. If the \textsc{entity} is a \textsc{player}, the
  sum also includes the \textsc{share price} of each owned
  \textsc{share}.
\item[Company] Represented by company cards. Companies start the game
  face-down in the \textsc{deck}. Eventually, each company in the
  \textsc{deck} is drawn and revealed. Newly drawn companies are
  \emph{unavailable} at first but become \emph{available} (for
  \textsc{auctions}) as instructed by the rules. A company is
  \emph{unowned} until \textsc{auctioned} to a \textsc{player} (in
  phase 1) or sold to the \textsc{foreign investor} (in phase 2),
  which turns it into a \emph{private company}, owned by the
  \textsc{player} or the \textsc{foreign investor}, respectively. It
  may later be bought by a \textsc{corporation} (in phase 3) or used
  to \textsc{form a corporation} (in phase 9). In both cases, it
  becomes a \emph{subsidiary company}. After that, it can be traded
  between \textsc{corporations} (in a later phase 3) but can never
  become a private company again.
\item[Corporation] Distinguished by their symbol. A corporation owns
  \emph{one} or more \textsc{companies} (\emph{never} zero), zero or
  more \$, and a symbol card showing its symbol. Stacked on top of
  that card are the unissued \textsc{shares} of the corporation. A
  corporation usually owns a share price card defining its
  \textsc{share price}. Depending on the \textsc{corporation}, there
  are 4 to 7 \textsc{shares} assigned to each corporation. At most 8
  corporations may exist simultaneously.
\item[Cost of ownership] Defined by the top card of the
  \textsc{deck}. Deducted from the income of \emph{each} company
  matching any of the colors in the central rectangle of the top
  card. If the top card is the game end card, a match with any color
  displayed by the card triggers the cost.
\item[Deck] Contains the unrevealed \textsc{company} cards and (at the
  bottom) the game end card. The back of the top company card displays
  the current \textsc{cost of ownership}. Once all companies have been
  drawn, the deck contains only the game end card for the rest of the
  game. The game end card is never drawn (but flipped eventually), and
  whichever face is up defines the \textsc{cost of ownership}.
\item[Entity] Each \textsc{player}, each \textsc{corporation}, the
  \textsc{foreign investor}, and the \textsc{bank} is an
  entity. Entities own the various assets (\textsc{money},
  \textsc{shares}, \textsc{companies}), which may only
  be transferred between entities according to the rules. Arrange the
  game components on the table in a way that clearly marks ownership.
  All assets are open for inspection by any player at any time.
\item[Face value] A \textsc{company} has a unique face value printed
  on its card.
\item[Face value order] Defines the order of \textsc{companies}.  The
  \textsc{company} with the highest \textsc{face value} is first in
  \emph{descending} face value order, but last in \emph{ascending}
  face value order.
\item[Foreign investor] An \textsc{entity} whose actions are entirely
  determined by the rules. Owns zero or more \$ and zero or more
  \textsc{companies}.
\item[Money] Measured in units of \$. Only integer \$ amounts of money
  are possible.
\item[Player] Owns zero or more \$, zero or more \textsc{companies}
  and zero or more \textsc{shares}. May become the \textsc{president}
  of any number of \textsc{corporations}.
\item[Player order] Marked by player order cards, starting with card
  \emph{1}, followed by the other cards in ascending order. However,
  the player order is cyclic and infinite. The player with the highest
  player order card is followed by the player with card \emph{1}, and
  the player order starts over.
\item[President] The president of a \textsc{corporation} is the
  \textsc{player} owning the \textsc{president's} \textsc{share} of that
  \textsc{corporation}. That player acts on behalf of the
  \textsc{corporation}. A \textsc{change of presidency} may occur
  during share trading. If the \textsc{bank} owns the \textsc{president's}
  \textsc{share}, the \textsc{corporation} is in
  \textsc{receivership}.
\item[Receivership] If all \emph{issued} \textsc{shares} of a
  \textsc{corporation} are owned by the \textsc{bank}, the
  \textsc{corporation} is in receivership. Its actions are entirely
  determined by the rules until receivership ends, which happens if a
  player buys a \textsc{share} of the \textsc{corporation} from the
  bank. The bought \textsc{share} is automatically the \textsc{president's}
  \textsc{share}, and the buying player becomes the new
  \textsc{president}. Duringe receivership, place a receivership card
  next to the \textsc{corporations's} assets as a reminder.
\item[Row] The row of share price cards. For practical reasons, often
  laid out in multiple rows. Logically, it is still one long row of
  share price cards in ascending order, from left to right. When a
  share price card is taken by a \textsc{corporation}, do not move
  other cards to fill the gap. Whenever a share price card is
  returned to the row, return it to its old spot.
\item[Share] Represented by share cards featuring the symbol of the
  corporation they are assigned to. \emph{Issued} shares are owned by
  either \textsc{players} or the \textsc{bank}. \emph{Unissued} shares
  are placed on top of the symbol card of the corresponding
  \textsc{corporation}. Shares are numbered. Numbers are only
  significant for unissued shares (as the the number of issued shares
  needs to be tracked). If a share is removed from the stack of
  unissued shares, always take the top one. The share marked
  \emph{President} is called the \emph{president's share}.
\item[Share price] Current value of each of the \textsc{shares} of a
  \textsc{corporation}. Usually marked by a share price card placed
  next to the other assets of the \textsc{corporation}. A
  \textsc{corporation} without a share price card has a share price of
  \$75.
\item[Share price order] Defines the order of
  \textsc{corporations}. Always descending, i.\,e. highest
  \textsc{share price} first.
\item[Synergy] A bonus income for \textsc{corporations} (only). For
  each pair of \textsc{companies} owned by the same
  \textsc{corporation} that have each other's code printed in one of
  their synergy blocks, add the \$ amount printed in the upper left
  corner of that block. Add this amount only \emph{once} per
  pair. Place a matching synergy marker on \emph{one} of the
  \textsc{companies} of each pair to facilitate counting.
\end{my_description}

\subsection*{Procedures}

Procedures are atomic. If any part of a procedure cannot be executed,
the whole procedure cannot be executed.

\begin{my_description}
\item[Adjust share price] On the share price card of the acting
  \textsc{corporation}, find the number of \emph{required} stars
  corresponding to the number of issued \textsc{shares}. Calculate the
  number of \emph{owned} stars by summing up all stars printed on
  \textsc{companies} owned by the \textsc{corporation} and adding
  another star for every \$10 \textsc{money} owned by the
  \textsc{corporation}. The \emph{Star} \textsc{corporation} adds
  additional two stars to the result. If the numbers of
  \emph{required} and \emph{owned} stars are equal, the \textsc{share
    price} does not change and the procedure is complete. If the
  number of \emph{owned} stars is at least two lower, exactly one
  lower, exactly one higher, or at least two higher than the number of
  \emph{required} stars, then the target \textsc{share price} can be
  found on the share price card next to the double left arrow, the
  left arrow, the right arrow, or the double right arrow,
  respectively. Take the matching share price card. If it is not
  available, take the next available share price card in the direction
  of the arrow. Return the old share price card to the row. If you end
  up taking the \$0 share price card, continue with \textsc{going
    bankrupt}. If you have to take the next higher available share
  price card but there is no higher share price card available, do not
  take any share price card.
\item[Auction] A \textsc{player} starting an auction chooses one of
  the \emph{available} \textsc{companies} and bids at least its
  \textsc{face value}. In \textsc{player order} (following the
  \textsc{player} that started the auction), \textsc{players} may
  raise the bid or leave the auction. \textsc{Players} that have left
  the auction are skipped for the remainder of that same auction. Bids
  are \$ amounts and may not exceed the \textsc{money} owned by the
  bidding \textsc{player}. Once all but one \textsc{player} have left
  the auction, the remaining \textsc{player} pays their bid to the
  \textsc{bank} and receives the \textsc{company}. Draw a new
  \textsc{company} from the \textsc{deck} and turn it vertically. It
  is \emph{unavailable} for later auctions during the same
  phase. After the auction, the next regular action of phase 1 is
  taken by the next \textsc{player} in \textsc{player order} following
  the \textsc{player} that \emph{started} the auction.
\item[Buy one share] \textsc{Shares} may be bought from the
  \textsc{bank}. The acting \textsc{player} takes one \textsc{share}
  from the \textsc{bank}. (If the corresponding \textsc{corporation} is
  in \textsc{receivership}, the \textsc{share} must be the
  \textsc{president's} \textsc{share}.) The corresponding
  \textsc{corporation} returns its share price card to the
  \textsc{row} and gains the next higher available share price
  card. The \textsc{player} pays the \emph{new} \textsc{share price}
  to the \textsc{bank}. Check for \textsc{change of presidency}. If
  the new \textsc{share price} is \$75, the game ends immediately
  after payment.
\item[Change of presidency] If at any time one or more
  \textsc{players} own more \textsc{shares} of any given
  \textsc{corporation} than the current \textsc{president} of that
  \textsc{corporation}, the next \textsc{player} in \textsc{player
    order} (after the current \textsc{president}) that owns more
  \textsc{shares} of that \textsc{corporation} than the current
  \textsc{president} becomes the new \textsc{president} and exchanges
  one of their own shares of that \textsc{corporation} with the
  \textsc{president's} \textsc{share} of that \textsc{corporation}.
\item[Collect income] Each \textsc{company} has an income printed on
  the company card. This income is reduced by the \textsc{cost of
    ownership} of that \textsc{company}. Each \textsc{entity} adds up
  the results for all \textsc{companies} it owns. The \textsc{foreign
    investor} also adds \$5. A \textsc{corporation} also adds the
  \textsc{synergy} between its \textsc{companies}. Some
  \textsc{corporations} benefit from additional income bonuses:
  \emph{Jupiter} adds \$1 per \textsc{company} it owns; \emph{Saturn}
  doubles the printed income if its best \textsc{company};
  \emph{Horse} adds \$1 for every two of its synergy markers;
  \emph{Bear} reduces its total \textsc{cost of ownership} by up to
  \$10 (the \textsc{cost of ownership} cannot become negative). If the
  final result is positive, the \textsc{entity} receives that amount
  from the \textsc{bank}. If the result is negative, it pays that
  amount to the \textsc{bank}. If a \textsc{corporation} is not able
  to do so, it \textsc{goes bankrupt}.
\item[Form corporation] Separate the acting \textsc{company} from the
  other assets of its owning \textsc{player} (as it is about to become
  a subsidiary \textsc{company} of a new \textsc{corporation}). The
  \textsc{player} selects an unused symbol card and places it above
  the \textsc{company}. Place the corresponding sorted stack of
  \textsc{shares} on top of the symbol card. The \textsc{player}
  selects an available share price card that features the color of the
  \textsc{company} in the \emph{IPO} rectangle and places it to the
  left of the symbol card. The \textsc{player} receives the
  \textsc{president's} \textsc{share} from the top of the stack. The
  \textsc{bank} receives the 2nd share from the stack. If the
  \textsc{face value} of the \textsc{company} is greater than the
  selected \textsc{share price}, both the \textsc{player} and the
  \textsc{bank} receive an additional share from the stack. Then, the
  \textsc{player} pays the difference between the summed up
  \textsc{share price} of the \textsc{shares} they have received and
  the \textsc{face value} of the \textsc{company} to the newly formed
  \textsc{corporation}. The \textsc{bank} pays the total summed up
  \textsc{share price} of the \textsc{shares} it has received to the
  newly formed \textsc{corporation}.
\item[Go bankrupt] Remove all \textsc{companies} of the acting
  \textsc{corporation} from the game. Set \emph{all} \textsc{shares}
  assigned to the \textsc{corporation} aside in a sorted stack on top
  of the \textsc{corporations's} symbol card, available for forming a
  new corporation later in the game. Return the \textsc{money} of the
  \textsc{corporation} to the \textsc{bank}. Return its share price
  card to the \textsc{row}. Set aside the receivership card
  if the \textsc{corporation} was in \textsc{receivership}.
\item[Issue one share] Works in the same way as \textsc{selling one
    share}, with the following exceptions: Replace the acting
  \textsc{player} by the acting \textsc{corporation}, and take the
  share given to the \textsc{bank} from the stack of unissued
  \textsc{shares} of the acting \textsc{corporation}. If the acting
  \textsc{corporation} is \emph{Eagle}, the \textsc{share price} does
  not change. The \emph{Eagle} \textsc{corporation} simply receives
  the current \textsc{share price} for the \textsc{share} it issues.
\item[Pay dividends] From the money it owns, the acting
  \textsc{corporation} pays dividends for each \emph{issued}
  \textsc{share} to the \textsc{entity} owning the share. The payout
  per share must be greater or equal \$0 and may not exceed the
  maximum amount as stated on the share price card of the
  \textsc{corporation}.
\item[Sell one share] The acting \textsc{player} gives one owned
  \textsc{share} to the \textsc{bank}. That \textsc{share} may only be
  the \textsc{president's} \textsc{share} if it is the
  \textsc{player's} last \textsc{share} of the corresponding
  \textsc{corporation}. The corresponding \textsc{corporation} returns
  its share price card to the \textsc{row} and gains the next lower
  available share price card. The \textsc{bank} pays the \emph{new}
  \textsc{share price} to the \textsc{player}. If the new
  \textsc{share price} is \$0, the \textsc{corporation} \textsc{goes
    bankrupt}. If the sold \textsc{share} was the last \textsc{share}
  of the same \textsc{corporation} owned by any \textsc{player}, the
  \textsc{corporation} goes into \textsc{receivership}. Otherwise,
  check for \textsc{change of presidency}. If the \textsc{president's}
  \textsc{share} was sold, consider the acting \textsc{player} the
  current \textsc{president} while checking for \textsc{change of
    presidency}.
\end{my_description}

}

\end{multicols}
\end{document}

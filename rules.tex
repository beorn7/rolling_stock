\documentclass[11pt,a4paper]{article}
\usepackage[utf8]{inputenc}
\usepackage{multicol,anysize}
\usepackage{sfmath,mathabx,wasysym}

%\renewcommand{\familydefault}{\sfdefault}

%\papersize{21.0cm}{29.7cm} 
\marginsize{1.0cm}{1.0cm}{1.0cm}{1.0cm}
\setlength{\parindent}{0pt}
\setlength{\parskip}{0.3ex plus 0.5ex minus 0.2ex}

\newcounter{itemcounter}
\newenvironment{my_enumerate}
{\begin{list}{\arabic{itemcounter}.}
  {\usecounter{itemcounter}\leftmargin=1.8em}
  \setlength{\itemsep}{1pt}
  \setlength{\parskip}{0pt}
  \setlength{\parsep}{0pt}
}
{\end{list}}

\newenvironment{my_description}
  {\begin{list}{}{\setlength{\labelwidth}{0pt}
   \leftmargin=1.2em
   \setlength{\itemindent}{-\leftmargin}
   %\setlength{\listparindent}{\parindent}
   \renewcommand{\makelabel}{\descriptionlabel}}
  \setlength{\itemsep}{1pt}
  \setlength{\parskip}{0pt}
  \setlength{\parsep}{0pt}
  }
  {\end{list}}

\newenvironment{my_itemize}
{\begin{list}{\labelitemi}{\leftmargin=1.2em}
  \setlength{\itemsep}{1pt}
  \setlength{\parskip}{0pt}
  \setlength{\parsep}{0pt}}
{\end{list}}


\begin{document}

\begin{center}
\section*{Rolling Stock}

A card game for three to five players. Rules v1.2pre. Game design by
Björn Rabenstein.
\end{center}

These are the complete and canonical rules. However, they are meant as
a reference and the ultimate ``source of truth'' and are written in an
extremely condensed and formal way. They should \emph{not} be used to
learn the rules. Read \emph{Learning the Game} instead.  Read the
\emph{Player's Guide} for additional information about the game
(strategy, variants, \dots).

\begin{multicols}{2}

{
%\small

  Terms printed \textsc{in small caps} are explained in the
  \emph{Glossary} or in the \emph{Procedures} section at the end of
  the rules.

\subsection*{Components}

The game contains 196 cards, 5 turn summaries, 3 booklets (the
\emph{Rules}, \emph{Learning the Game}, and the \emph{Player's
  Guide}), an unlimited number of round double-sided synergy markers,
and an unlimited amount of \textsc{money}. (In practice, the number of
synergy markers and amount of \textsc{money} provided with the game is
obviously limited. Use other means of tracking them in the unlikely
case that the material provided runs out.) The 196 cards in more
detail:
\begin{my_itemize}
\item 10 symbol cards, each featuring a different
  \textsc{corporation} symbol.
\item 100 \textsc{shares}, 10 for each of the 10
  \textsc{corporations}, identified by their symbols.
\item 5 grey player order cards, numbered from 1 to 5.
\item 32 white share price cards showing \textsc{share prices} from
  \$0 to \$100 and a table used for \textsc{adjusting share
    prices}. The cards are two-sided. Use the back for
  \textsc{corporations} with 7 or more \textsc{shares} issued. The
  share price cards from \$10 to \$45 have a colored \emph{IPO}
  rectangle to mark \textsc{share prices} that can be used to
  \textsc{form a corporation}.
\item 45 \textsc{companies}, 6 red ($\bullet$), 8 orange
  ($\blacktriangleup$), 8 yellow ($\sqbullet$), 7 green ($\pentagon$),
  8 blue ($\hexagon$), 8 purple ($\star$). (Each geometric shape is
  used to mark the respective color on all game components to assist
  color-blind players.) The number in the upper right corner is the
  income. The number in the upper left corner is the \textsc{face
    value}, followed by an allowed price span (in parentheses) used
  for \textsc{company} trading in phase 6. The \textsc{face value} is
  unique for each \textsc{company}. Other unique identifiers are a one
  to five letter code and the full \textsc{company} name, to be found
  in the center of the top half of the card. The bottom half of each
  card shows blocks of \textsc{synergies} with other
  \textsc{companies}.
\item 3 game end cards, one each for the \emph{training game}, the
  \emph{short game}, and the \emph{long game}.
\item 1 grey \textsc{foreign investor} card.
\end{my_itemize}

The normal orientation of cards and money is called
\emph{horizontal}. In many situations during the game, cards or money
are turned \emph{vertically} to mark a special state of that
particular component.


\subsection*{Setup}

Each \textsc{player} receives \$30 from the \textsc{bank}, a turn
summary, and a random player order card, thus determining the initial
\textsc{player order}. Set aside the \textsc{shares} in 10 separate
stacks, one for each \textsc{corporation}, sorted with the
1\textsuperscript{st} share on top. Add the corresponding symbol card to
the bottom of each stack. Those stacks are available to \textsc{form
  corporations} in phase 2. Lay out the share price cards in a
\textsc{row}, in ascending order from left to right. Place the
\textsc{foreign investor} card on the table. The \textsc{foreign
  investor} receives \$4 from the \textsc{bank}. Build the
\textsc{deck} of \textsc{companies} in the following way: Place the
game end card that corresponds to your desired game type (training
game, short game, full game) at the bottom of the deck, the lower
cost-of-ownership side up. Separate the \textsc{companies} by
color. Shuffle each stack. Draw (without revealing) one more
\textsc{company} than number of \textsc{players} from each stack, with
the following exceptions: In a 4-player game, draw 6 orange
\textsc{companies}; in a 5-player game, draw 8 orange
\textsc{companies}. Discard all \textsc{companies} not drawn (without
revealing them). Add the purple \textsc{companies} to the deck on top
of the game end card if you play a full game (otherwise discard all
purple \textsc{companies}). Then add the blue \textsc{companies} to
the top of the deck if you play a full or a short game (otherwise,
discard all blue \textsc{companies}). Then add the green
\textsc{companies} to the top of the deck, then the yellow
\textsc{companies}, then the orange \textsc{companies}, and finally
the red \textsc{companies}. Place the deck in the middle of the
table. Draw and reveal a number of \textsc{companies} equal to the
number of \textsc{players}. The drawn \textsc{companies} are available
for future \textsc{auctions}. Start the first turn of the game.

\subsection*{Game Sequence}

The game is played in a series of turns. Each turn is divided into the
following phases (in that order):

\begin{my_enumerate}
\item In \textsc{share price order}, each \textsc{corporation}
  may \textsc{issue one share}. Then it turns its share price card
  horizontally (whether it has issued a share or not).
\item In descending \textsc{face value order}, private companies may
  \textsc{form corporations}.
\item In \textsc{player order}, players choose \emph{one} of the
  following actions: \textsc{buy one share}, \textsc{sell one share},
  start an \textsc{auction}, or pass. After a player has passed, their
  player order card is turned vertically. After each non-pass action,
  it is turned horizontally. As soon as all player order cards are
  turned vertically, the phase ends.
\item Determine the new \textsc{player order} by descending remaining
  cash. Ties are broken by the old \textsc{player order}. Turn all
  player order cards horizontally and redistribute them between
  players to indicate the new \textsc{player order}.
\item In \emph{ascending} \textsc{face value order}, the
  \textsc{foreign investor} buys as many of the remaining
  \emph{available} \textsc{companies} as possible, paying \textsc{face
    value}. For each \textsc{company} bought in that way, draw a new
  one from the \textsc{deck} and turn it vertically. Once the
  \textsc{foreign investor} cannot buy any more \textsc{companies},
  all \emph{unavailable} \textsc{companies} become \emph{available}
  for future \textsc{auctions}. (Turn their cards horizontally.)
\item In \textsc{any order}, \textsc{corporations} buy
  \textsc{companies} from other \textsc{corporations},
  \textsc{players}, or the \textsc{foreign investor}. Each transaction
  is executed separately and involves exactly one \textsc{company},
  which is transfered from the seller to the buying
  \textsc{corporation}, and the \textsc{money} payed, which is
  transfered from the buying \textsc{corporation} to the seller. A
  transaction only takes place if both \textsc{entities} involved
  agree on a price within the allowed price span printed on the
  company card. The transfered \textsc{company} and \textsc{money}
  are turned vertically until the end of this phase and cannot be used
  for further transactions. (Turn them back horizontally after the
  phase has ended. A \textsc{company} turned vertically may be the
  only \textsc{company} a \textsc{corporation} owns.) The
  \textsc{foreign investor} only sells at the maximum allowed price,
  and the intention to buy a specific \textsc{company} from the
  \textsc{foreign investor} must be announced before executing the
  transaction. Any \textsc{corporation} with a higher \textsc{share
    price} and enough \textsc{money} available may then immediately
  intervene and buy the desired \textsc{company} itself. If more than
  one eligible \textsc{corporation} wants to intervene, the one with
  the highest \textsc{share price} has priority. If there is no
  intervention, the announcing \textsc{corporation} \emph{must} buy
  the specified \textsc{company}.
\item In \textsc{any order}, \textsc{players} and
  \textsc{corporations} may close \textsc{companies} by removing them
  from the game. \textsc{Players} must close a sufficient number of
  \textsc{companies} with negative income to be able to pay for a
  possibly negative total income in phase 8. The \textsc{foreign investor}
  closes \textsc{companies} whose \textsc{cost of ownership} exceeds
  their income.
\item \textsc{Players}, \textsc{corporations}, and the \textsc{foreign
    investor} \textsc{collect income}.
\item In \textsc{share price order}, \textsc{corporations} \textsc{pay
    dividends}, \textsc{adjust their share price}, and turn their
  share price card vertically.
\item If there are no unowned \textsc{companies} left, flip the game
  end card. If it is already flipped, or if the \$100 share price card
  is owned by a \textsc{corporation}, the game ends.
\end{my_enumerate}

Repeat the turn sequence until the game ends in phase 3 or phase
10. Then rank players according to their \textsc{book value},
breaking ties by \textsc{player order}.

\subsection*{Glossary}

\begin{my_description}
\item[Any order] During a phase executed in any order, players act (for
  themselves or as \textsc{presidents} on behalf of
  \textsc{corporations}) whenever and as often as they wish, even
  concurrently or alternating with other players. They may wait for
  other players' actions but when nobody has acted for a reasonable
  amount of time, the phase ends.
\item[Bank] An \textsc{entity} whose actions are entirely determined
  by the rules. Owns an unlimited amount of \textsc{money} and all
  \emph{issued} \textsc{shares} that are not owned by
  \textsc{players}.
\item[Book value] The sum of the \textsc{face value} of each
  \textsc{company} and the \textsc{money} owned by an
  \textsc{entity}. If the \textsc{entity} is a \textsc{player}, the
  sum also includes the \textsc{share price} of each owned
  \textsc{share}.
\item[Company] Represented by company cards. Companies start the game
  face-down in the \textsc{deck}. Eventually, each company in the
  \textsc{deck} is drawn and revealed. Newly drawn companies are
  \emph{unavailable} at first but become \emph{available} (for
  \textsc{auctions}) as instructed by the rules. A company is
  \emph{unowned} until \textsc{auctioned} to a \textsc{player} (in
  phase 3) or sold to the \textsc{foreign investor} (in phase 5),
  which turns it into a \emph{private company}, owned by the
  \textsc{player} or the \textsc{foreign investor}, respectively. It
  may later be bought by a \textsc{corporation} (in phase 6) or used
  to \textsc{form a corporation} (in phase 2). In both cases, it
  becomes a \emph{subsidiary company}. After that, it can be traded
  between \textsc{corporations} (in a later phase 6) but can never
  become a private company again.
\item[Corporation] Distinguished by their symbol. A corporation owns
  \emph{one} or more \textsc{companies} (\emph{never} zero), zero or
  more \$, and a card showing its symbol. Stacked on top of that card
  are the unissued \textsc{shares} of the corporation. A corporation
  usually owns a share price card defining its \textsc{share
    price}. There are 10 \textsc{shares} assigned to each
  corporation. At most 10 corporations may exist simultaneously.
\item[Cost of ownership] Defined by the top card of the
  \textsc{deck}. Deducted from the income of \emph{each} company
  matching any of the colors in the central rectangle of the top
  card. If the top card is the game end card, a match with any color
  displayed by the card triggers the cost.
\item[Deck] Contains the unrevealed \textsc{company} cards and (at the
  bottom) the game end card. The back of the top company card displays
  the current \textsc{cost of ownership}. Once all companies have been
  drawn, the deck contains only the game end card for the rest of the
  game. The game end card is never drawn (but flipped eventually), and
  whichever face is up defines the \textsc{cost of ownership}.
\item[Entity] Each \textsc{player}, each \textsc{corporation}, the
  \textsc{foreign investor}, and the \textsc{bank} is an
  entity. Entities own the various assets (\textsc{money},
  \textsc{shares}, \textsc{companies}), which may only
  be transferred between entities according to the rules. Arrange the
  game components on the table in a way that clearly marks ownership.
  All assets are open for inspection by any player at any time.
\item[Face value] A \textsc{company} has a unique face value printed
  on its card.
\item[Face value order] Defines the order of \textsc{companies}.  The
  \textsc{company} with the highest \textsc{face value} is first in
  \emph{descending} face value order, but last in \emph{ascending}
  face value order.
\item[Foreign investor] An \textsc{entity} whose actions are entirely
  determined by the rules. Owns zero or more \$ and zero or more
  \textsc{companies}.
\item[Money] Measured in units of \$. Only integer \$ amounts of money
  are possible.
\item[Player] Owns zero or more \$, zero or more \textsc{companies}
  and zero or more \textsc{shares}. May become the \textsc{president}
  of any number of \textsc{corporations}.
\item[Player order] Marked by player order cards, starting with card
  \emph{1}, followed by the other cards in ascending order. However,
  the player order is cyclic and infinite. The player with the highest
  player order card is followed by the player with card \emph{1}, and
  the player order starts over.
\item[President] The president of a \textsc{corporation} is the
  \textsc{player} owning the \emph{president's} \textsc{share} of that
  \textsc{corporation}. That player acts on behalf of the
  \textsc{corporation}. A \textsc{change of presidency} may occur
  during share trading.
\item[Row] The row of share price cards. For practical reasons, often
  laid out in multiple rows. Logically, it is still one long row of
  share price cards in ascending order, from left to right. When a
  share price card is taken by a \textsc{corporation}, do not move
  other cards to fill the gap. Whenever a share price card is
  returned to the row, return it to its old spot.
\item[Share] Represented by share cards featuring the symbol of the
  corporation they are assigned to. \emph{Issued} shares are owned by
  either \textsc{players} or the \textsc{bank}. \emph{Unissued} shares
  are placed next to the other assets of the corresponding
  \textsc{corporation}. Shares are numbered from 1 to 10. Numbers are
  only significant for unissued shares (as they display the number of
  issued shares). If a share is removed from the stack of unissued
  shares, always take the top one. The share marked \emph{President}
  is called the \emph{president's share}.
\item[Share price] Current value of each of the \textsc{shares} of a
  \textsc{corporation}. Usually marked by a share price card placed
  next to the other assets of the \textsc{corporation}. A
  \textsc{corporation} without a share price card has a share price of
  \$100.
\item[Share price order] Defines the order of
  \textsc{corporations}. Always descending, i.\,e. highest
  \textsc{share price} first.
\item[Synergy] A bonus income for \textsc{corporations} (only). For
  each pair of \textsc{companies} owned by the same
  \textsc{corporation} that have each other's code printed in one of
  their synergy blocks, add the \$ amount printed in the upper left
  corner of that block. Add this amount only \emph{once} per
  pair. Place a matching synergy marker on \emph{one} of the
  \textsc{companies} of each pair to facilitate counting.

\end{my_description}

\subsection*{Procedures}

Procedures are atomic. If any part of a procedure cannot be executed,
the whole procedure cannot be executed.

\begin{my_description}
\item[Adjust share price] On the share price card of the acting
  \textsc{corporation}, choose the column corresponding to the number
  of issued \textsc{shares}. In that column, find the \$ interval that
  matches the \textsc{book value} of the \textsc{corporation}. From
  there, go left to the beginning of the row to find the target
  \textsc{share price}. Take the matching share price card. If it is
  not available, take the next available share price card in the
  direction of the arrow printed next to the target \textsc{share
    price}. Return the old share price card to the row. If you end up
  taking the \$0 share price card, continue with \textsc{going
    bankrupt}. If you have to take the next higher available share
  price card but there is no higher share price card available, do not
  take any share price card.
\item[Auction] A \textsc{player} starting an auction chooses one of
  the \emph{available} \textsc{companies} and bids at least its
  \textsc{face value}. In \textsc{player order} (following the
  \textsc{player} that started the auction), \textsc{players} may
  raise the bid or leave the auction. \textsc{Players} that have left
  the auction are skipped for the remainder of that same auction. Bids
  are \$ amounts and may not exceed the \textsc{money} owned by the
  bidding \textsc{player}. Once all but one \textsc{player} have left
  the auction, the remaining \textsc{player} pays their bid to the
  \textsc{bank} and receives the \textsc{company}. Draw a new
  \textsc{company} from the \textsc{deck} and turn it vertically. It
  is \emph{unavailable} for later auctions during the same
  phase. After the auction, the next regular action of phase 3 is
  taken by the next \textsc{player} in \textsc{player order} following
  the \textsc{player} that \emph{started} the auction.
\item[Buy one share] \textsc{Shares} may be bought from the
  \textsc{bank}. The acting \textsc{player} takes one \textsc{share}
  from the \textsc{bank}. The corresponding \textsc{corporation}
  returns its share price card to the \textsc{row} and gains the next
  higher available share price card. The \textsc{player} pays the
  \emph{new} \textsc{share price} to the \textsc{bank}. Check for
  \textsc{change of presidency}. If the new \textsc{share price} is
  \$100, the game ends immediately after payment.
\item[Change of presidency] If at any time one or more
  \textsc{players} own more \textsc{shares} of any given
  \textsc{corporation} than the current \textsc{president} of that
  \textsc{corporation}, the latter exchanges the \emph{president's}
  \textsc{share} of that \textsc{corporation} with a \textsc{share} of
  the same \textsc{corporation} owned by the next \textsc{player} in
  \textsc{player order} (after the current \textsc{president})
  that owns more \textsc{shares} of that \textsc{corporation} than
  the current \textsc{president}.
\item[Collect income] Each \textsc{company} has an income printed on
  the company card. This income is reduced by the \textsc{cost of
    ownership} of that \textsc{company}. Each \textsc{entity} adds up
  the results for all \textsc{companies} it owns. A
  \textsc{corporation} also adds the \textsc{synergy} between its
  \textsc{companies}. The \textsc{foreign investor} adds \$5. If the
  result is positive, the \textsc{entity} receives that amount from
  the \textsc{bank}. If the result is negative, it pays that amount to
  the \textsc{bank}. If a \textsc{corporation} is not able to do so,
  it \textsc{goes bankrupt}.
\item[Form corporation] Separate the acting \textsc{company} from the
  other assets of its owning \textsc{player} (as it is about to become
  a subsidiary \textsc{company} of a new \textsc{corporation}). The
  \textsc{player} selects an unused stack of \textsc{shares} and an
  available share price card that features the color of the
  \textsc{company} in the \emph{IPO} rectangle and places both next to
  the \textsc{company}. The \textsc{player} takes the smallest
  possible number of \textsc{shares} from the stack so that their
  added \textsc{share price} matches or exceeds the \textsc{face
    value} of the \textsc{company}, and pays the difference between
  added \textsc{share price} and \textsc{face value} to the newly
  formed \textsc{corporation}. Then take the same number of
  \textsc{shares} from the stack (again) and give them to the
  \textsc{bank}. The \textsc{bank} pays their added \textsc{share
    price} to the newly formed \textsc{corporation}.
\item[Go bankrupt] Remove all \textsc{companies} of the acting
  \textsc{corporation} from the game. Set \emph{all} 10
  \textsc{shares} assigned to the \textsc{corporation} aside in a
  sorted stack. (These \textsc{shares} are available for forming a new
  corporation later.) Return its \textsc{money} to the \textsc{bank}. Return
  its share price card to the \textsc{row}.
\item[Issue one share] Works in the same way as \textsc{selling one
    share}, with the following exceptions: Replace the acting
  \textsc{player} by the acting \textsc{corporation}, and take the
  share given to the \textsc{bank} from the stack of unissued
  \textsc{shares} of the acting \textsc{corporation}.
\item[Pay dividends] From the money it owns, the acting
  \textsc{corporation} pays dividends for each \emph{issued}
  \textsc{share} to the \textsc{entity} owning the share. The payout
  per share must be greater or equal \$0 and may not exceed the
  maximum amount as stated on the share price card of the
  \textsc{corporation}.
\item[Sell one share] A \textsc{share} may be sold to the
  \textsc{bank} if there is at least one other
  \textsc{share} of the same \textsc{corporation} owned by any
  \textsc{player}.  The acting \textsc{player} gives one owned
  \textsc{share} to the \textsc{bank}. The corresponding
  \textsc{corporation} returns its share price card to the
  \textsc{row} and gains the next lower available share price
  card. The \textsc{bank} pays the \emph{new} \textsc{share price} to
  the \textsc{player}. If the new \textsc{share price} is \$0, the
  \textsc{corporation} \textsc{goes bankrupt}. Check for
  \textsc{change of presidency}.
\end{my_description}

}

\end{multicols}
\end{document}
